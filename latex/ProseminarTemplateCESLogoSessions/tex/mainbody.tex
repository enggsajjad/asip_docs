\chapter*{REMOTE ACCESS -TUTORIAL}
\section*{X2GO \& MobaXterm, Public Key Authentication and .bashrc
	Settings}
\subsection{X2Go Installation}
\begin{enumerate}
	\item
	X2Go is program used to run graphical applications on our Linux
	machines remotely. This uses a different technology from remote X,
	which results in better performance, especially when not on campus.
	This also allows for suspending and resuming sessions and programs,
	while they continue to run. This allows the use of long-running
	graphical applications.
	\item
	The X2Go Client software is already installed on all lab computers.
	For your personal computers, download the X2Go client for your
	operating system:
		\begin{enumerate}[label*=\arabic*.]
		\item macOS - 		\url{http://code.x2go.org/releases/X2GoClient_latest_macosx.dmg}
		\item Windows - \url{http://code.x2go.org/releases/X2GoClient_latest_mswin32-setup.exe}
	\end{enumerate}
	\item
	On macOS, mount the dmg file (your browser may do this for you after
	download) and drag the x2goclient application to your Applications
	folder. On Windows, double-click the downloaded setup file and follow
	the instructions in the wizard to install the software.
\end{enumerate}
\begin{figure}
	\centering
	\includegraphics[width=4.28968in,height=4.17836in]{src/images/image1.png}
	\caption{Session Tab}
	\label{fig:fig1}
\end{figure}
\subsection{X2Go Configuration}
\begin{enumerate}[resume]
	\item
	When you first run the X2Go client, you will be presented with a
	"\emph{\textbf{New session}}" dialog. You should fill this in with
	this information:
	\item
	Session tab
		\begin{enumerate}[label*=\arabic*.]
		\item
		Session name - Any name you'd like to identify the session to
		yourself - if you're connecting to
		\emph{\textbf{i80labpc04.ira.uka.de}}, you might just want this to
		be " \emph{\textbf{i80labpc04}}"
		\item
		Host - Full name of the server you're connecting to, e.g.
		\emph{\textbf{i80labpc04.ira.uka.de}}
		\item
		All of our compute machines have the X2Go server installed
		\item
		Login -- Your user ID e.g. ``\emph{\textbf{asip04}}'' (be careful to
		use lower case)
		\item
		Session type - Select XFCE (this is the only supported session type)
		- see below for of session types.
	\end{enumerate}
	\item
	Connection tab
	\begin{enumerate}[label*=\arabic*.]
		\item
		Connection speed - Set the connection speed you will most often use
		for this connection
		\item
		The default ADSL is fine for most connections, but if you are on
		campus, you will get better graphics performance if you choose LAN
	\end{enumerate}
	\item
	Input/output tab
	\begin{enumerate}[label*=\arabic*.]
		\item
		Display - select whether you want to run full screen or at a
		specific resolution
	\end{enumerate}
	\item
	Media
	\begin{enumerate}[label*=\arabic*.]
		\def\labelenumii{\arabic{enumii}.}
		\item
		Client side printing support - be sure to uncheck this box or you
		may get errors when starting the session
	\end{enumerate}
\end{enumerate}
\begin{figure}
	\centering
	\includegraphics[width=4.66535in,height=5.27165in]{src/images/image2.png}
	\caption{Connection Tab}
	\label{fig:fig2}
\end{figure}
\begin{figure}
	\centering
	\includegraphics[width=4.00394in,height=4.5315in]{src/images/image3.png}
	\caption{Input/output Tab}
	\label{fig:fig3}
\end{figure}
\begin{figure}
	\centering
	\includegraphics[width=4.00394in,height=4.5315in]{src/images/image4.png}
	\caption{Media Tab}
	\label{fig:fig4}
\end{figure}
\begin{figure}
	\centering
	\includegraphics[width=5.50787in,height=3.38189in]{src/images/image5.png}
	\caption{Main Client Screen}
	\label{fig:fig5}
\end{figure}
\subsection{X2Go Session Types}
\begin{enumerate}
	\def\labelenumi{\arabic{enumi}.}
	\setcounter{enumi}{8}
	\item
	The session types that we support are:
	
	\begin{enumerate}
		\def\labelenumii{\arabic{enumii}.}
		\item
		XFCE (recommended) - This is a low-power window manager that is the
		only one supported in the current version of Ubuntu
		\item
		Published applications - This allows you to run one or more
		applications directly, rather than a full desktop session. See below
		for how to run published applications
	\end{enumerate}
\end{enumerate}

\begin{enumerate}
	\def\labelenumi{\Alph{enumi}.}
	\setcounter{enumi}{3}
	\item
	\textbf{X2Go Connecting}
\end{enumerate}

\begin{enumerate}
	\def\labelenumi{\arabic{enumi}.}
	\setcounter{enumi}{9}
	\item
	To start the session, click on it and provide your password where
	prompted
\end{enumerate}
\begin{figure}
	\centering
	\includegraphics[width=5.50787in,height=3.38189in]{src/images/image6.png}
	\caption{Login Screen}
	\label{fig:fig6}
\end{figure}
\begin{enumerate}
	\def\labelenumi{\arabic{enumi}.}
	\setcounter{enumi}{10}
	\item
	After you click OK, it will connect to the server and start your
	session. Watch the Status line to see what's happening. Once the
	status is "running," your session should launch.
	\item
	When you first connect to a particular server, you may get a dialog
	box asking you to accept the host key. Click Yes to accept it:
\end{enumerate}
\begin{figure}
	\centering
	\includegraphics[width=3.58289in,height=1.45815in]{src/images/image7.png}
	\caption{Host Key Authentication}
	\label{fig:fig6}
\end{figure}
\begin{enumerate}
	\def\labelenumi{\arabic{enumi}.}
	\setcounter{enumi}{12}
	\item
	To suspend a session, click the suspend button:
\end{enumerate}
\begin{figure}
	\centering
	\includegraphics[width=6.08268in,height=3.71654in]{src/images/image8.png}
	\caption{Suspending a Session}
	\label{fig:fig8}
\end{figure}
\begin{enumerate}
	\def\labelenumi{\arabic{enumi}.}
	\setcounter{enumi}{13}
	\item
	To terminate, either log out of your session or click the terminate
	button:
\end{enumerate}
\begin{figure}
	\centering
	\includegraphics[width=6.26806in,height=3.82311in]{src/images/image9.png}
	\caption{Terminating a Session}
	\label{fig:fig9}
\end{figure}
\begin{enumerate}
	\def\labelenumi{\Alph{enumi}.}
	\setcounter{enumi}{4}
	\item
	\textbf{X2Go Resuming}
\end{enumerate}

\begin{enumerate}
	\def\labelenumi{\arabic{enumi}.}
	\setcounter{enumi}{14}
	\item
	If you have a single session open on a particular server and you
	reconnect with the same client, it will automatically re-connect to
	your session.
	\item
	If you are connecting from a different client or have multiple
	sessions on the same server, you'll be presented a list to either
	resume an existing session or create a new one:
\end{enumerate}
\begin{figure}
	\centering
	\includegraphics[width=6.26806in,height=3.8057in]{src/images/image10.png}
	\caption{Resuming a Session}
	\label{fig:fig10}
\end{figure}
\begin{enumerate}
	\def\labelenumi{\Alph{enumi}.}
	\setcounter{enumi}{5}
	\item
	\textbf{X2Go Published Applications}
\end{enumerate}

\begin{enumerate}
	\def\labelenumi{\arabic{enumi}.}
	\setcounter{enumi}{16}
	\item
	If you choose the "Published Applications" session, after you connect
	the Status will change to running, but it will appear that nothing has
	happened. To choose an application, click the "Applications" button:
\end{enumerate}
\begin{figure}
	\centering
	\includegraphics[width=6.26806in,height=3.83718in]{src/images/image11.png}
	\caption{Published Applications}
	\label{fig:fig11}
\end{figure}
\begin{enumerate}
	\def\labelenumi{\arabic{enumi}.}
	\setcounter{enumi}{17}
	\item
	This will then bring up a dialog from which you can choose an
	application to run. All Thayer-specific applications are interspersed
	under the "Other" section:
\end{enumerate}
\begin{figure}
	\centering
	\includegraphics[width=4.07153in,height=3.96413in]{src/images/image12.png}
	\caption{Applications}
	\label{fig:fig12}
\end{figure}
\begin{enumerate}
	\def\labelenumi{\arabic{enumi}.}
	\setcounter{enumi}{18}
	\item
	When you click "Start," be patient. This dialog does not go away and
	some programs may take several seconds to start up. Clicking Start
	more than once will launch multiple instances of the same app.
	\item
	Fonts for Windows
	\item
	When using the X2Go client on Windows, there may be some older
	programs (e.g. Cadence) where fonts do not show up properly. Symptoms
	you may see are either blocky, illegible fonts or instances where
	fonts disappear because they are white on a white background. If you
	experience any of these, you can install a font package that should
	eliminate most of these issues. Keep in mind that this is only needed
	if you are running the X2Go client on Windows.
	\item
	First, if you haven't already, follow the instructions at Thayer
	Shares Connecting to connect to Thayer Shares. Navigate to the Courses
	share (P:), and go to the software\textbackslash x2go folder. In this
	folder, double-click the vcxsrv\_fonts.exe file to install the fonts.
	Depending on your security settings, you may need to drag this file to
	your local computer before double-clicking on it.
	\item
	Other Settings
	\item
	If you are using a Mac and need to use the Alt key within remote
	sessions, you need to change the X11 preferences. Run XQuartz directly
	from within Applications-\textgreater Utilities. Then, select the
	X11-\textgreater Preferences... menu item, select the Input tab, and
	check the box next to "Option keys send Alt\_L and Alt\_R." Close the
	preferences window and quit X11. Then, restart X2Go and when you log
	into a remote session, the option key (also labeled alt on most Mac
	keyboards) will send the Alt key to the remote side.
\end{enumerate}
\begin{figure}
	\centering
	\includegraphics[width=3.78346in,height=2.57874in]{src/images/image13.png}
	\caption{X11 Preferences}
	\label{fig:fig13}
\end{figure}
\textbf{\hfill\break
}

\textbf{MobaXterm CLIENT -TUTORIAL}

\begin{enumerate}
	\def\labelenumi{\Alph{enumi}.}
	\setcounter{enumi}{6}
	\item
	\textbf{MobaXterm Installation}
\end{enumerate}

\begin{enumerate}
	\def\labelenumi{\arabic{enumi}.}
	\item
	Download MobaXterm from
	\url{https://mobaxterm.mobatek.net/download.html}
	\item
	Install it with default settings.
\end{enumerate}

\begin{enumerate}
	\def\labelenumi{\Alph{enumi}.}
	\setcounter{enumi}{7}
	\item
	\textbf{MobaXterm Configuration}
\end{enumerate}

\begin{enumerate}
	\def\labelenumi{\arabic{enumi}.}
	\setcounter{enumi}{2}
	\item
	Click on "Sessions" and then on "SSH".
\end{enumerate}
\begin{figure}
	\centering
	\includegraphics[width=5.73515in,height=3.25226in]{src/images/image14.JPG}
	\caption{SSH Session}
	\label{fig:fig14}
\end{figure}
\begin{enumerate}
	\def\labelenumi{\arabic{enumi}.}
	\setcounter{enumi}{3}
	\item
	In the new windows "Session settings", enter \textbf{Remote host} as
	``\emph{i80labpcXX.ira.uka.de}'', tick the box "\textbf{Specify user
		name}" and then enter your user name as ``\emph{asip-abcdnn}''.
\end{enumerate}
\begin{figure}
	\centering
	\includegraphics[width=5.53861in,height=3.70864in]{src/images/image15.JPG}
	\caption{Session Settings}
	\label{fig:fig15}
\end{figure}
\begin{enumerate}
	\def\labelenumi{\arabic{enumi}.}
	\setcounter{enumi}{4}
	\item
	Press OK. It will ask for the user password.
	\item
	Enter your password and press Enter.
\end{enumerate}
\begin{figure}
	\centering
	\includegraphics[width=5.67963in,height=3.28535in]{src/images/image16.JPG}
	\caption{Logging in and Requiring Password}
	\label{fig:fig16}
\end{figure}
\begin{enumerate}
	\def\labelenumi{\arabic{enumi}.}
	\setcounter{enumi}{6}
	\item
	You are now logged into the lab PC using MobaXterm.
\end{enumerate}
\begin{figure}
	\centering
	\includegraphics[width=5.70556in,height=3.27527in]{src/images/image17.JPG}
	\caption{Logged into Remote PC}
	\label{fig:fig17}
\end{figure}
\begin{enumerate}
	\def\labelenumi{\Alph{enumi}.}
	\setcounter{enumi}{8}
	\item
	\textbf{Recommended Practice.}
\end{enumerate}

\begin{enumerate}
	\def\labelenumi{\arabic{enumi}.}
	\setcounter{enumi}{7}
	\item
	It is \textbf{recommended} to log into \textbf{i80pc57} as this PC
	contains the ASIPmeister software. Try to perform lab on this PC. Use
	\textbf{i80labpc10} when you need to implement your applications on
	FPGA.
	\item
	To repeatedly login to some PC and avoid password, use DSA-keys and
	copy to desired PC.
	\item
	Type "\textbf{ssh-keygen -t dsa}" and press Enter. Leave the default
	options. Leave the password empty.
\end{enumerate}

asip-sajjad04@i80labpc09:\textasciitilde:\$ssh-keygen -t dsa

Generating public/private dsa key pair.

Enter file in which to save the key (/home/asip-sajjad04/.ssh/id\_dsa):

/home/asip-sajjad04/.ssh/id\_dsa already exists.

Overwrite (y/n)? y

Enter passphrase (empty for no passphrase):

Enter same passphrase again:

Your identification has been saved in /home/asip-sajjad04/.ssh/id\_dsa.

Your public key has been saved in /home/asip-sajjad04/.ssh/id\_dsa.pub.

The key fingerprint is:

af:c3:84:62:4e:f4:e6:5e:cb:d1:03:19:ff:63:a9:ad asip-sajjad04@i80labpc09

The key's randomart image is:

+-\/-{[} DSA 1024{]}-\/-\/-\/-+

\textbar{} \textbar{}

\textbar{} \textbar{}

\textbar{} . \textbar{}

\textbar{} . + \textbar{}

\textbar{} . . .S . \textbar{}

\textbar{} + + .+ . . \textbar{}

\textbar{} + + oo + = \textbar{}

\textbar{} . .oo+ = . \textbar{}

\textbar{} .. +.E.. \textbar{}

+-\/-\/-\/-\/-\/-\/-\/-\/-\/-\/-\/-\/-\/-\/-\/-\/-+

\begin{enumerate}
	\def\labelenumi{\arabic{enumi}.}
	\setcounter{enumi}{10}
	\item
	Then copy this generated DSA-key to desired PC by type following
	command and enter your password.
\end{enumerate}

asip-sajjad04@i80labpc09:\textasciitilde:\$ssh-copy-id -i
\textasciitilde/.ssh/id\_dsa.pub asip-sajjad04@i80pc57

/usr/bin/ssh-copy-id: INFO: attempting to log in with the new key(s), to
filter out any that are already installed

/usr/bin/ssh-copy-id: INFO: 1 key(s) remain to be installed -\/- if you
are prompted now it is to install the new keys

asip-sajjad04@i80pc57's password:

Number of key(s) added: 1

Now try logging into the machine, with: "ssh 'asip-sajjad04@i80pc57'"

and check to make sure that only the key(s) you wanted were added.

\begin{enumerate}
	\def\labelenumi{\arabic{enumi}.}
	\setcounter{enumi}{11}
	\item
	Now log into i80pc57 using ``ssh -X'' it will ask for the password.
\end{enumerate}

asip-sajjad04@i80labpc09:\textasciitilde:\$ssh -X i80pc57

Last login: Wed May 6 05:11:07 2020 from i80labpc09.irf.uni-karlsruhe.de

asip-sajjad04@i80pc57:\textasciitilde:\$

\begin{enumerate}
	\def\labelenumi{\Alph{enumi}.}
	\setcounter{enumi}{9}
	\item
	\textbf{Setting up .bashrc.user}
\end{enumerate}

\begin{enumerate}
	\def\labelenumi{\arabic{enumi}.}
	\setcounter{enumi}{12}
	\item
	Whenever you are logged into any PC, this file is executed at the
	login. Please set different variables in this file carefully. Usually
	the following variables should be like this:
\end{enumerate}

asip-sajjad04@i80pc57:\textasciitilde:\$cat .bashrc.user

export ASIPS\_LICENSE=29000@i80asip.ira.uka.de

export PATH=/AM/ASIPmeister/bin:\$PATH

export ASIP\_APDEV\_SRCROOT=/home/asip00/epp/AM\_tools

export PATH=/usr/java/jre1.6.0\_45/bin:\$PATH

export ASIPmeister\_Home=/AM/ASIPmeister

export ASIPmeister\_HOME=/AM/ASIPmeister

. /home/adm/modelsim\_66d.setup

. /home/adm/xilinx\_13.2\_32bit.setup

asip-sajjad04@i80pc57:\textasciitilde:\$
