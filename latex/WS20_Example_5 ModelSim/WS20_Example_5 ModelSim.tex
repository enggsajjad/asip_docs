% Options for packages loaded elsewhere
\PassOptionsToPackage{unicode}{hyperref}
\PassOptionsToPackage{hyphens}{url}
%
\documentclass[
]{article}
\usepackage{amsmath,amssymb}
\usepackage{lmodern}
\usepackage{iftex}
\ifPDFTeX
  \usepackage[T1]{fontenc}
  \usepackage[utf8]{inputenc}
  \usepackage{textcomp} % provide euro and other symbols
\else % if luatex or xetex
  \usepackage{unicode-math}
  \defaultfontfeatures{Scale=MatchLowercase}
  \defaultfontfeatures[\rmfamily]{Ligatures=TeX,Scale=1}
\fi
% Use upquote if available, for straight quotes in verbatim environments
\IfFileExists{upquote.sty}{\usepackage{upquote}}{}
\IfFileExists{microtype.sty}{% use microtype if available
  \usepackage[]{microtype}
  \UseMicrotypeSet[protrusion]{basicmath} % disable protrusion for tt fonts
}{}
\makeatletter
\@ifundefined{KOMAClassName}{% if non-KOMA class
  \IfFileExists{parskip.sty}{%
    \usepackage{parskip}
  }{% else
    \setlength{\parindent}{0pt}
    \setlength{\parskip}{6pt plus 2pt minus 1pt}}
}{% if KOMA class
  \KOMAoptions{parskip=half}}
\makeatother
\usepackage{xcolor}
\IfFileExists{xurl.sty}{\usepackage{xurl}}{} % add URL line breaks if available
\IfFileExists{bookmark.sty}{\usepackage{bookmark}}{\usepackage{hyperref}}
\hypersetup{
  hidelinks,
  pdfcreator={LaTeX via pandoc}}
\urlstyle{same} % disable monospaced font for URLs
\setlength{\emergencystretch}{3em} % prevent overfull lines
\providecommand{\tightlist}{%
  \setlength{\itemsep}{0pt}\setlength{\parskip}{0pt}}
\setcounter{secnumdepth}{-\maxdimen} % remove section numbering
\ifLuaTeX
  \usepackage{selnolig}  % disable illegal ligatures
\fi

\author{}
\date{}

\begin{document}

\textbf{MODELSIM SIMULATION -TUTORIAL}

\begin{enumerate}
\def\labelenumi{\Alph{enumi}.}
\item
  \textbf{Setting up your first ASIP project directory structure}
\end{enumerate}

\begin{enumerate}
\def\labelenumi{\arabic{enumi}.}
\item
  Login to any \emph{\textbf{i80labpcXX.ira.uka.de}} directly or using
  SSH or using X2Go Client. For example, login as
  \emph{\textbf{asip-sajjad04}} into
  \emph{\textbf{i80labpc02.ira.uka.de}}
\item
  Open shell terminal from the start menu. It should be in your default
  home directory. Go to the directory
  ``\emph{\textbf{\textasciitilde/ASIP\_SS17/Session1/ASIPMeisterProjects/brownie:\$}}''
\item
  Set the proper path and parameters in ``env\_settings'' like dlxsim
  path, project path and project name.
\item
  Go to the application directory, for example:
  ``\emph{\textbf{\textasciitilde/ASIP\_SS17/Session1/ASIPMeisterProjects/brownie/Applications/Arith:\$}}''
  and type ``\emph{\textbf{make clean}}'' clean this directory it there
  are previously generated files.
\item
  Generate the Compiler using ASIPmeister if you have a C application.
\item
  Compile the C application using ``\emph{\textbf{make sim}}''. A
  directory ``\emph{\textbf{BUILD\_SIM}}'' is created which contains
  different temporary files and a .dlxsim file to be simulated in dlxsim
  (in this case it is ``\emph{\textbf{Arith.dlxsim}}''). In this
  directory, the files ``\emph{\textbf{TestData.IM}}'' and
  ``\emph{\textbf{TestData.DM}}''are the file used during the ModelSim
  simulation.
\item
  Simulate your application in dlxsim simulator using
  ``\emph{\textbf{make dlxsim}}'', just to verify the functionality.
\item
  Go to the directory
  ``\emph{\textbf{\textasciitilde/ASIP\_SS17/Session1/ASIPMeisterProjects/brownie:\$}}''
  Open the ASIPmeister project, modify the CPU if required, and generate
  VHDL files for simulation/synthesis and files for compiler generation.
\end{enumerate}

asip04@i80labpc04:\textasciitilde/ASIP\_SS17/Session1/ASIPMeisterProjects/brownie:\$ASIPmeister
brownie.pdb \&

\begin{quote}
This will create a meister folder in current directory having three
subdirectories (brownie.sim, brownie.syn, brownie.sw) and some
architectural and description files. This will also generate the
compiler and assembler for the CPU.
\end{quote}

\begin{enumerate}
\def\labelenumi{\arabic{enumi}.}
\setcounter{enumi}{8}
\item
  Go to the directory
  ``\emph{\textbf{\textasciitilde/ASIP\_SS17/Session1/ASIPMeisterProjects/brownie/ModelSim:\$}}''
  Start the ModelSim using ``\emph{\textbf{vsim}}''
\end{enumerate}

asip04@i80labpc04:\textasciitilde/ASIP\_SS17/Session1/ASIPMeisterProjects/brownie/ModelSim:\$vsim
\&

\begin{enumerate}
\def\labelenumi{\arabic{enumi}.}
\setcounter{enumi}{9}
\item
  If ModelSim asks for ``modelsim.ini'' choose the default one like
  ``/Software/ModelSim/ModelSim\_6.6d/modeltech/modelsim.ini''
\item
  Open File Menu \textgreater{} New \textgreater{} Project and enter a
  project name and change the project location to the ModelSim directory
  in your project directory. Confirm the dialog with the OK button.
\item
  Choose ``Add Existing File'' button and browse to the
  ``meister/brownie.syn'' directory of your ASIP Meister project and
  select all the VHDL files for synthesis.
\item
  Again, choose ``Add Existing File'' button and add the testbench
  files: ``tb\_browstd32.vhd'', ``MemoryMapperTypes.vhd'',
  ``MemoryMapper.vhd'', and ``Helper.vhd'' from the ModelSim directory
  of your current project.
\item
  {[}Optional{]} Configure the CPU Frequency for which you want to
  simulate your CPU, default is 50 MHz Open the ModelSim testbench
  (``tb\_ browstd32.vhd''), search for CLK\_PERIOD, and change the value
  accordingly in ``ns''.
\item
  Compile the project using Compile Menu \textgreater{} Compile Order
  \textgreater{} Auto Generate. Every file should have a green mark
  behind its name, showing that the compilation was successful.
\item
  Run the simulation using Simulate Menu \textgreater{} Start
  Simulation. Open the work library, mark the entry
  ``\emph{\textbf{cfg}}'' (that is the VHDL configuration for the
  testbench) in the list and press OK. That will start the simulation
  and you will get another two tabs attached to the Workspace window
  (sim / Files).
\item
  To load some predefined simulation settings, choose Tools Menu
  \textgreater{} Tcl \textgreater{} Execute Macro and select the
  ``\emph{wave\_vhdl.do}'' file in your ModelSim directory and press OK
  to load it. The wave-window is filled with certain signals that are
  useful to evaluate the simulation of the program execution on the
  processor.
\item
  {[}Optional{]} If you want to dump VCD file of your design for power
  estimation, you can enter following commands in ModelSim command
  prompt:
\end{enumerate}

VSIM \textgreater{} vsim -t 1ns work.cfg

VSIM \textgreater{} vcd file test.vcd

VSIM \textgreater{} vcd add -r test/dut/*

\begin{enumerate}
\def\labelenumi{\arabic{enumi}.}
\setcounter{enumi}{18}
\item
  Press the button ``\emph{\textbf{Run all}}'' to run the simulation
  until it aborts. At the end of a simulation the message ``Failure:
  Simulation End'' is printed to show successful end of simulation. At
  the simulation end, the file ``\emph{\textbf{TestData.OUT}}'' is
  created in your ModelSim directory. It contains the content of the
  simulated memory after the CPU finished working. Therefore, if your
  algorithm is storing the result in the memory you can find the values
  here.
\end{enumerate}

\end{document}
