\chapter*{ASIPMEISTER -TUTORIAL}
\section*{Creating ASIPmeister Project}
\subsection{Environement Settings}
\begin{enumerate}
\item Login to any \emph{\textbf{i80labpcXX.ira.uka.de}} directly or using
	SSH or using X2Go Client. For example, login as \emph{\textbf{asip-sajjad04}} into
	\emph{\textbf{i80labpc02.ira.uka.de}}
\item Open shell terminal from the start menu. It should be in your default
	home directory. Type ``\emph{\textbf{pwd}}''
\begin{lstlisting}
asip04@i80labpc04:~:$pwd
/home/asip04
\end{lstlisting}
\item Create a new directory for your Lab e.g., ``\emph{\textbf{ASIP\_SS17''}}
\begin{lstlisting}
asip04@i80labpc04:~:$mkdir ASIP_SS17
asip04@i80labpc04:~:$cd ASIP_SS17/
\end{lstlisting}
\item Create a new directory for your lab session in
	``\emph{\textbf{ASIP\_SS17''}} e.g. ``\emph{\textbf{Session1''}}
\begin{lstlisting}
asip04@i80labpc04:~/ASIP_SS17:$mkdir Session1
asip04@i80labpc04:~/ASIP_SS17:$cd Session1/
\end{lstlisting}
\item Create a directory for your ASIP project in ``\emph{\textbf{Session1''}}
\begin{lstlisting}
asip04@i80labpc04:~/ASIP_SS17/Session1:$mkdir
ASIPMeisterProjects
asip04@i80labpc04:~/ASIP_SS17/Session1:$cd
ASIPMeisterProjects/
\end{lstlisting}
\item For each ASIPmeister CPU create a separate in
	``\emph{\textbf{ASIPMeisterProjects''}}. For example, copy template
	project and rename it e.g. ``\emph{\textbf{brownie''}} for ASIPmeister
	CPU ``\emph{\textbf{brownie.pdb}}''
\begin{lstlisting}
asip04@i80labpc04:~/ASIP_SS17/Session1/ASIPMeisterProjects:$cp
-r /home/asip00/epp/ASIPMeisterProjects/TEMPLATE_PROJECT ./brownie
asip04@i80labpc04:~/ASIP_SS17/Session1/ASIPMeisterProjects:$ls
brownie
\end{lstlisting}
\item Set the parameters and settings of the ASIP project in
	``\emph{\textbf{env\_settings}}''
\item Open ASIPMeister CPU in the respective directory i.e. in brownie
\begin{lstlisting}
asip04@i80labpc04:~/ASIP_SS17/Session1/ASIPMeisterProjects/brownie:$ASIPmeister
brownie.pdb &
\end{lstlisting}
\item Modify the CPU in ASIPmeister and generate the required files. A
	``\emph{\textbf{meister}}'' directory will be created in your ASIP
	project directory i.e. in ``\emph{\textbf{brownie}}''
\item For another CPU copy template project and rename it e.g.
	``\emph{\textbf{brownieOPT''}} for ASIPmeister CPU
	``\emph{\textbf{brownieSPEED.pdb}}''
\begin{lstlisting}
asip04@i80labpc04:~/ASIP_SS17/Session1/ASIPMeisterProjects:$cp
-r /home/asip00/ASIPMeisterProjects/TEMPLATE\_PROJECT ./brownieOPT
asip04@i80labpc04:~/ASIP_SS17/Session1/ASIPMeisterProjects:$ls
brownie brownieOPT
\end{lstlisting}
\item Set the parameters and settings of the ASIP project in
	``\emph{\textbf{env\_settings}}''
\item Open ASIPMeister CPU in the respective directory i.e. in brownieOPT
\begin{lstlisting}
asip04@i80labpc04:~/ASIP_SS17/Session1/ASIPMeisterProjects/brownieOPT:$ASIPmeister
brownieSPEED.pdb &
\end{lstlisting}
\item Modify the CPU in ASIPmeister and generate the required files. A
	``\emph{\textbf{meister}}'' directory will be created in your ASIP
	project directory i.e. in ``\emph{\textbf{brownieOPT}}''
\end{enumerate}