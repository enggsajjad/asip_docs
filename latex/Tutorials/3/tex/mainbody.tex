\chapter*{DLX SIMULATOR -TUTORIAL}
\section*{Simulation using DLXSim}
\subsection{Simulating an Assembly file}
\begin{enumerate}
	\item Login to any \emph{\textbf{i80labpcXX.ira.uka.de}} directly or using
	SSH or using X2Go Client. For example login as \emph{\textbf{asip-sajjad04}} into
	\emph{\textbf{i80labpc02.ira.uka.de}}
	\item Open shell terminal from the start menu. It should be in your default
	home directory. Go to the directory
	``\emph{\textbf{~/ASIP\_SS17/Session1/ASIPMeisterProjects/brownie:\$}}''
	\item Set the proper path and parameters in ``env\_settings'' like dlxsim
	path, project path and project name.
	\item Go to the application directory, for example:
	``\emph{\textbf{~/ASIP\_SS17/Session1/ASIPMeisterProjects/brownie/Applications/Arith:\$}}''
	and type ``\emph{\textbf{make clean}}'' clean this directory it there
	are previously generated files.
\begin{lstlisting}
asip04@i80labpc04:~/ASIP_SS17/Session1/ASIPMeisterProjects/brownie/Applications/Arith:$make clean
/bin/rm -rf BUILD_SIM BUILD_FPGA
asip04@i80labpc04:~/ASIP_SS17/Session1/ASIPMeisterProjects/brownie/Applications/Arith:$ls
1\_Arith.s Makefile
asip04@i80labpc04:~/ASIP_SS17/Session1/ASIPMeisterProjects/brownie/Applications/Arith:$
\end{lstlisting}
	\item As this application subdirectory contains .s file, you can directly
	simulate it using ``\emph{\textbf{make dlxsim}}'' without compiling
	it. If this application has .c file, then you have to compile it using
	``\emph{\textbf{make sim}}''. For example to load
	``\emph{\textbf{1\_Arith.s}}'' and using no forwarding, use the
	following parameters. A directory ``\emph{\textbf{BUILD\_SIM}}'' is
	created which contains different temporary files and a .dlxsim file to
	be simulated in dlxsim (in this case it is
	``\emph{\textbf{Arith.dlxsim}}'').
\begin{lstlisting}
asip04@i80labpc04:~/ASIP_SS17/Session1/ASIPMeisterProjects/brownie/Applications/Arith:$make dlxsim DLXSIM_PARAM="-f1_Arith.s -da0 -pf0"
-----------------------------------------------
Transforming file "1_Arith.s" for target SIMULATION.
-----------------------------------------------
-----------------------------------------------
Assembling/Linking for target SIMULATION:
-----------------------------------------------
Creating combined files.
STACK_START:			0xFFFFC
-----------------------------------------------
FINISHED ASSEMBLING/LINKING for target SIMULATION.
-----------------------------------------------


-----------------------------------------------
Starting dlxsim:
-----------------------------------------------
/Software/epp/dlxsim_Laboratory/dlxsim -fBUILD_SIM/Arith.dlxsim -f1_Arith.s -da0 -pf0
Biggest used address for Text Section (word aligned): 0x1c
Biggest used address for Data Section (word aligned): 0x0
(dlxsim)
\end{lstlisting}
	\item Then in dlxsim you can use ``go'' or ``step'' command to simulate all
	instructions or each instruction step by step respectively.
\begin{lstlisting}
(dlxsim) step
stopped after (single) step, pc = _main+0x04 (0x0004): addi r2,r0,0x9
(dlxsim) step
stopped after (single) step, pc = _main+0x08 (0x0008): or r3,r1,r2
(dlxsim) get r2
r2:	0x00000009
(dlxsim) go
TRAP #0 received
Altogether 41,0e0(41) cycles executed.
0 Warnings for unresolved data dependencies printed.
0 Warnings for successive load/store commands printed.
0 Warnings for load/stores in the text section printed.
(dlxsim) 
\end{lstlisting}
\item You can see different statistics using ``\emph{\textbf{stats}}''
	command.
\item Enter ``\emph{\textbf{quit}}'' command to exit from dlxsim simulator.
\end{enumerate}
\subsection{Simulating a C file}
\begin{enumerate}[resume]
\item If the application consists of C files then you can use
	``\emph{\textbf{make sim}}'', which will compile your application into
	assembly file and automatically starts dlxsim. The other steps remain
	the same. Remember, ``\emph{\textbf{make sim}}'' only works if you
	have already created Compiler.
\begin{lstlisting}
asip04@i80labpc04:~/ASIP_SS17/Session1/ASIPMeisterProjects/brownie/Applications/Arith:$make
sim
\end{lstlisting}
\item You can have different parameter to ``\emph{\textbf{make sim}}'' like
	optimization identifier and number of NOPS added for simulating your
	application in hardware.
\begin{lstlisting}
asip04@i80labpc04:~/ASIP_SS17/Session1/ASIPMeisterProjects/brownie/Applications/Arith:$make
sim GCC\_PARAM=-O3
\end{lstlisting}
\item	You can now start dlxsim simulation using following different
	commands:
\begin{lstlisting}
asip04@i80labpc04:~/ASIP_SS17/Session1/ASIPMeisterProjects/brownie/Applications/Arith:$make dlxsim GCC_PARAM=-O3
\end{lstlisting}
OR
\begin{lstlisting}
asip04@i80labpc04:~/ASIP_SS17/Session1/ASIPMeisterProjects/brownie/Applications/Arith:$make
dlxsim DLXSIM_PARAM="-fBUILD_SIM/Arith.dlxsim -da0 --pf1"
\end{lstlisting}
\color{red}
\item \normalcolor You can save dlxsim simulation output to different file using
	``\emph{\textbf{--lf}}'', ``\emph{\textbf{--uf}}'', or
	\emph{\textbf{``--af}}'' for LCD, UART or audio respectively as
	following:
\begin{lstlisting}
asip04@i80labpc04:~/ASIP_SS17/Session1/ASIPMeisterProjects/brownie/Applications/Arith:$make
dlxsim DLXSIM_PARAM="-fBUILD_SIM/Arith.dlxsim -da0
--pf1 --lfoutput_dlxsim.txt"
\end{lstlisting}
\end{enumerate}