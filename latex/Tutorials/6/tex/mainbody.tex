\chapter*{XILINX ISE -TUTORIAL}
\section*{Synthesis and Implementation}
\subsection{Xilinx ISE Framework for Hardware Implementation}
\begin{enumerate}
	\item
	Login to any \emph{\textbf{i80labpcXX.ira.uka.de}} directly or using
	SSH or using X2Go Client. For example, login as
	\emph{\textbf{asip-sajjad04}} into
	\emph{\textbf{i80labpc02.ira.uka.de}}
	\item
	Open shell terminal from the start menu. It should be in your default
	home directory. Go to the project directory 
	``\emph{\small{\textbf{\textasciitilde/ASIP\_SS17/Session1/ASIPMeisterProjects/brownie:\$}}}``
	\item
	Set the proper path and parameters in ``env\_settings'' like dlxsim
	path, project path and project name.
	\item
	Go to an application directory for example 
	``\emph{\small{\textbf{\textasciitilde/ASIP\_SS17/Session1/ASIPMeisterProjects/brownie/Applications/Arith:\$}}}``
	and type ``\emph{\textbf{make clean}}'' clean this directory it there
	are previously generated files.
	\item
	Generate the VHDL files and Compiler if you have not done yet.
	\item
	Compile the C application using ``\emph{\textbf{make sim}}''.
	\item
	Simulate your application in dlxsim simulator using
	``\emph{\textbf{make dlxsim}}'', just to verify the functionality.
	\item
	Go to the directory 
	``\emph{\small{\textbf{\textasciitilde/ASIP\_SS17/Session1/ASIPMeisterProjects/brownie:\$}}}''
	Open the ASIPmeister project, modify the CPU if required, and generate
	VHDL files for simulation/synthesis and files for compiler generation.
	\item
	Go to the directory 
	``\emph{\small{\textbf{\textasciitilde/ASIP\_SS17/Session1/ASIPMeisterProjects/brownie/ModelSim:\$}}}''
	Simulate the design in ModelSim to verify hardware simulation.
	\item
	Go to the directory 
	``\emph{\small{\textbf{\textasciitilde/ASIP\_SS17/Session1/ASIPMeisterProjects/brownie:\$}}}''
	and type ``ise \&'' to start Xilinx ISE.
	\item Create new project using File Menu \textgreater{} New Project with
	following project settings:
\begin{lstlisting}
Project Name: ISE_Framework
Project Path:
home/asip04/ASIP_SS17/Session1/ASIPMeisterProjects/brownie/ISE_Framework
Device Family: Virtex5
Device: xc5vlx110t
Package: ff1136
\end{lstlisting}

	\item
	Add the design and framework files by selecting ``Project Menu
	\textgreater{} Add Copy of Sources'' then brows to:
	
	\begin{enumerate}
		\def\labelenumii{\alph{enumii}.}
		\item
		``\emph{\small{\textbf{\textasciitilde/ASIP\_SS17/Session1/ASIPMeisterProjects/brownie/ISE\_Framework}}}'' and select all the files
		\item
		``\emph{\small{\textbf{\textasciitilde/ASIP\_SS17/Session1/ASIPMeisterProjects/brownie/ISE\_Framework/IP-Cores}}}'' and select all the files
		\item
		``\emph{\small{\textbf{\textasciitilde/ASIP\_SS17/Session1/ASIPMeisterProjects/brownie/meister/brownie.syn}}}'' and select all the files
	\end{enumerate}
	\item
	Now you can synthesize, implement and generate programming file for
	the design using the following respectively:
	
	\begin{enumerate}
		\def\labelenumii{\alph{enumii}.}
		\item
		Processes Menu \textgreater{} Synthesize XST
		\item
		Processes Menu \textgreater{} Implement Design
		\item
		Processes Menu \textgreater{} Generate Programming File
	\end{enumerate}
	\item
	Once the design is implemented you can see different reports using:
	
	\begin{enumerate}
		\def\labelenumii{\alph{enumii}.}
		\item
		Processes Menu \textgreater{} Place \& Route \textgreater{} Generate
		Post Place \& Route Static Timing \textgreater{} Detailed Reports
		\textgreater{} Place and Route Report
		\item
		Processes Menu \textgreater{} Place \& Route \textgreater{} Generate
		Post Place \& Route Static Timing \textgreater{} Detailed Reports
		\textgreater{} Post PAR Static Timing Report
		\item
		Processes Menu \textgreater{} Place \& Route \textgreater{} Analyze
		Post Place \& Route Static Timing \textgreater{} Timing Constraints
	\end{enumerate}
	\item
	In the directory 
	``\emph{\small{\textbf{\textasciitilde/ASIP\_SS17/Session1/ASIPMeisterProjects/brownie/Applications/Arith:\$}}}''
	and type ``hterm \&'' to start HyperTerminal to see the UART output if
	there is any.
	\item
	In the directory 
	``\emph{\small{\textbf{\textasciitilde/ASIP\_SS17/Session1/ASIPMeisterProjects/brownie/Applications/Arith:\$}}}''
	and type ``make fpga'', it will combine the generate DM/IM file with
	your ISE generated bitstream. Finally, a new bitstream file contains
	your hardware CPU along with corresponding IM/DM files of your
	application will be generated in the folder ``BUILD\_FPGA''. This
	bitstream will be used to configure the FPGA.
	\item
	In the directory
	``\emph{\small{\textbf{\textasciitilde/ASIP\_SS17/Session1/ASIPMeisterProjects/brownie/Applications/Arith:\$}}}''
	type ``make upload'': to upload the existing bitstream to the FPGA
\end{enumerate}
\subsection{Xilinx ISE Framework for Benchmarking}
\begin{enumerate}[resume]
\item To accurately measure the critical path and area of the ASIPmeister CPU, you can use ISE\_Benchmark folder instead of ISE\_Framework folder.
\item Go to the directory ``\emph{\small{\textbf{\textasciitilde/ASIP\_SS17/Session1/ASIPMeisterProjects/brownie:\$}}}'' and type ``ise \&'' to start Xilinx ISE.
\item Create new project using File Menu \textgreater{} New Project with following project settings:
\begin{lstlisting}
Project Name: ISE_BenchMark
Project Path:
home/asip04/ASIP_SS17/Session1/ASIPMeisterProjects/brownie/ISE_BenchMark
Device Family: Virtex5
Device: xc5vlx110t
Package: ff1136
\end{lstlisting}
\item Add the design and framework files by selecting ``Project Menu \textgreater{} Add Copy of Sources'' then brows to:
	\begin{enumerate}
		\def\labelenumii{\alph{enumii}.}
		\item
		``\emph{\small{\textbf{\textasciitilde/ASIP\_SS17/Session1/ASIPMeisterProjects/brownie/
				ISE\_}}} \emph{\textbf{BenchMark}}'' and select all the files
		\item
		``\emph{\small{\textbf{\textasciitilde/ASIP\_SS17/Session1/ASIPMeisterProjects/brownie/
				meister/brownie.syn}}}'' and select all the files
	\end{enumerate}
\item Now you can synthesize, implement and generate programming file for the design as before.
\item Once the design is implemented you can see different reports as before.
\end{enumerate}
\subsection{Xilinx ISE Framework for XPower Power Estimation}
\begin{enumerate}[resume]
\item To accurately measure the power consumption of the ASIPmeister CPU, you can create another folder ISE\_XPower.
\item Go to the directory ``\emph{\small{\textbf{\textasciitilde/ASIP\_SS17/Session1/ASIPMeisterProjects/brownie:\$}}}'' and type ``ise \&'' to start Xilinx ISE.
\item Create new project using File Menu \textgreater{} New Project with following project settings:
\begin{lstlisting}
Project Name: ISE_XPower
Project Path:
home/asip04/ASIP_SS17/Session1/ASIPMeisterProjects/brownie/ISE_XPower
Device Family: Virtex5
Device: xc5vlx110t
Package: ff1136
\end{lstlisting}
\item Add only design files by selecting ``Project Menu \textgreater{} Add Copy of Sources'' then brows to ``\emph{\small{\textbf{\textasciitilde/ASIP\_SS17/Session1/ASIPMeisterProjects/brownie/meister/brownie.syn}}}'' and select all the files.
\item Now you can synthesize and implement the design as before.
\item Once the design is implemented you can open XPower tool using Processes Menu \textgreater{} Place \& Route \textgreater{} Analyze Power Distribution (XPower Analyzer)
\item Then in XPower Tool, select ``File Menu \textgreater{} OpenDesign'' and set the properties as follows:
	\begin{enumerate}
		\def\labelenumii{\alph{enumii}.}
		\item
		Design File:
		/home/asip04/ASIP\_SS17/Session1/ASIPMeisterProjects/brownie/ ISE\_
		XPower/brownie32.ncd
		\item
		Physical Constraint File:/
		home/asip04/ASIP\_SS17/Session1/ASIPMeisterProjects/brownie/ ISE\_
		XPower/ brownie32.pcf
		\item
		Simulation Activity File:/
		home/asip04/ASIP\_SS17/Session1/ASIPMeisterProjects/brownie/ModelSim/test.vcd
	\end{enumerate}
\item After analysing the activity file, the CPU power is estimated. You can see total and dynamic power of the FPGA. Also, you can confirm that the VCD file is loaded properly by verify the clock value in XPower.
\end{enumerate}

