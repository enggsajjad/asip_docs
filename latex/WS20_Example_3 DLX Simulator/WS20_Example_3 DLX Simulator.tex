% Options for packages loaded elsewhere
\PassOptionsToPackage{unicode}{hyperref}
\PassOptionsToPackage{hyphens}{url}
%
\documentclass[
]{article}
\usepackage{amsmath,amssymb}
\usepackage{lmodern}
\usepackage{iftex}
\ifPDFTeX
  \usepackage[T1]{fontenc}
  \usepackage[utf8]{inputenc}
  \usepackage{textcomp} % provide euro and other symbols
\else % if luatex or xetex
  \usepackage{unicode-math}
  \defaultfontfeatures{Scale=MatchLowercase}
  \defaultfontfeatures[\rmfamily]{Ligatures=TeX,Scale=1}
\fi
% Use upquote if available, for straight quotes in verbatim environments
\IfFileExists{upquote.sty}{\usepackage{upquote}}{}
\IfFileExists{microtype.sty}{% use microtype if available
  \usepackage[]{microtype}
  \UseMicrotypeSet[protrusion]{basicmath} % disable protrusion for tt fonts
}{}
\makeatletter
\@ifundefined{KOMAClassName}{% if non-KOMA class
  \IfFileExists{parskip.sty}{%
    \usepackage{parskip}
  }{% else
    \setlength{\parindent}{0pt}
    \setlength{\parskip}{6pt plus 2pt minus 1pt}}
}{% if KOMA class
  \KOMAoptions{parskip=half}}
\makeatother
\usepackage{xcolor}
\IfFileExists{xurl.sty}{\usepackage{xurl}}{} % add URL line breaks if available
\IfFileExists{bookmark.sty}{\usepackage{bookmark}}{\usepackage{hyperref}}
\hypersetup{
  hidelinks,
  pdfcreator={LaTeX via pandoc}}
\urlstyle{same} % disable monospaced font for URLs
\setlength{\emergencystretch}{3em} % prevent overfull lines
\providecommand{\tightlist}{%
  \setlength{\itemsep}{0pt}\setlength{\parskip}{0pt}}
\setcounter{secnumdepth}{-\maxdimen} % remove section numbering
\ifLuaTeX
  \usepackage{selnolig}  % disable illegal ligatures
\fi

\author{}
\date{}

\begin{document}

\textbf{DLX SIMULATOR -TUTORIAL}

\begin{enumerate}
\def\labelenumi{\Alph{enumi}.}
\item
  \textbf{Simulating an Assembly file}
\end{enumerate}

\begin{enumerate}
\def\labelenumi{\arabic{enumi}.}
\item
  Login to any \emph{\textbf{i80labpcXX.ira.uka.de}} directly or using
  SSH or using X2Go Client. For example login as
  \emph{\textbf{asip-sajjad04}} into
  \emph{\textbf{i80labpc02.ira.uka.de}}
\item
  Open shell terminal from the start menu. It should be in your default
  home directory. Go to the directory
  ``\emph{\textbf{\textasciitilde/ASIP\_SS17/Session1/ASIPMeisterProjects/brownie:\$}}''
\item
  Set the proper path and parameters in ``env\_settings'' like dlxsim
  path, project path and project name.
\item
  Go to the application directory, for example:
  ``\emph{\textbf{\textasciitilde/ASIP\_SS17/Session1/ASIPMeisterProjects/brownie/Applications/Arith:\$}}''
  and type ``\emph{\textbf{make clean}}'' clean this directory it there
  are previously generated files.
\end{enumerate}

asip04@i80labpc04:\textasciitilde/ASIP\_SS17/Session1/ASIPMeisterProjects/brownie/Applications/Arith:\$make
clean

/bin/rm -rf BUILD\_SIM BUILD\_FPGA

asip04@i80labpc04:\textasciitilde/ASIP\_SS17/Session1/ASIPMeisterProjects/brownie/Applications/Arith:\$ls

1\_Arith.s Makefile

asip04@i80labpc04:\textasciitilde/ASIP\_SS17/Session1/ASIPMeisterProjects/brownie/Applications/Arith:\$

\begin{enumerate}
\def\labelenumi{\arabic{enumi}.}
\setcounter{enumi}{4}
\item
  As this application subdirectory contains .s file, you can directly
  simulate it using ``\emph{\textbf{make dlxsim}}'' without compiling
  it. If this application has .c file, then you have to compile it using
  ``\emph{\textbf{make sim}}''. For example to load
  ``\emph{\textbf{1\_Arith.s}}'' and using no forwarding, use the
  following parameters. A directory ``\emph{\textbf{BUILD\_SIM}}'' is
  created which contains different temporary files and a .dlxsim file to
  be simulated in dlxsim (in this case it is
  ``\emph{\textbf{Arith.dlxsim}}'').
\end{enumerate}

asip04@i80labpc04:\textasciitilde/ASIP\_SS17/Session1/ASIPMeisterProjects/brownie/Applications/Arith:\$make
dlxsim DLXSIM\_PARAM="-f1\_Arith.s -da0 -pf0"

-\/-\/-\/-\/-\/-\/-\/-\/-\/-\/-\/-\/-\/-\/-\/-\/-\/-\/-\/-\/-\/-\/-\/-\/-\/-\/-\/-\/-\/-\/-\/-\/-\/-\/-\/-\/-\/-\/-\/-\/-\/-\/-\/-\/-\/-\/-

Transforming file "1\_Arith.s" for target SIMULATION.

-\/-\/-\/-\/-\/-\/-\/-\/-\/-\/-\/-\/-\/-\/-\/-\/-\/-\/-\/-\/-\/-\/-\/-\/-\/-\/-\/-\/-\/-\/-\/-\/-\/-\/-\/-\/-\/-\/-\/-\/-\/-\/-\/-\/-\/-\/-

-\/-\/-\/-\/-\/-\/-\/-\/-\/-\/-\/-\/-\/-\/-\/-\/-\/-\/-\/-\/-\/-\/-\/-\/-\/-\/-\/-\/-\/-\/-\/-\/-\/-\/-\/-\/-\/-\/-\/-\/-\/-\/-\/-\/-\/-\/-

Assembling/Linking for target SIMULATION:

-\/-\/-\/-\/-\/-\/-\/-\/-\/-\/-\/-\/-\/-\/-\/-\/-\/-\/-\/-\/-\/-\/-\/-\/-\/-\/-\/-\/-\/-\/-\/-\/-\/-\/-\/-\/-\/-\/-\/-\/-\/-\/-\/-\/-\/-\/-

Creating combined files.

STACK\_START: 0xFFFFC

-\/-\/-\/-\/-\/-\/-\/-\/-\/-\/-\/-\/-\/-\/-\/-\/-\/-\/-\/-\/-\/-\/-\/-\/-\/-\/-\/-\/-\/-\/-\/-\/-\/-\/-\/-\/-\/-\/-\/-\/-\/-\/-\/-\/-\/-\/-

FINISHED ASSEMBLING/LINKING for target SIMULATION.

-\/-\/-\/-\/-\/-\/-\/-\/-\/-\/-\/-\/-\/-\/-\/-\/-\/-\/-\/-\/-\/-\/-\/-\/-\/-\/-\/-\/-\/-\/-\/-\/-\/-\/-\/-\/-\/-\/-\/-\/-\/-\/-\/-\/-\/-\/-

-\/-\/-\/-\/-\/-\/-\/-\/-\/-\/-\/-\/-\/-\/-\/-\/-\/-\/-\/-\/-\/-\/-\/-\/-\/-\/-\/-\/-\/-\/-\/-\/-\/-\/-\/-\/-\/-\/-\/-\/-\/-\/-\/-\/-\/-\/-

Starting dlxsim:

-\/-\/-\/-\/-\/-\/-\/-\/-\/-\/-\/-\/-\/-\/-\/-\/-\/-\/-\/-\/-\/-\/-\/-\/-\/-\/-\/-\/-\/-\/-\/-\/-\/-\/-\/-\/-\/-\/-\/-\/-\/-\/-\/-\/-\/-\/-

/Software/epp/dlxsim\_Laboratory/dlxsim -fBUILD\_SIM/Arith.dlxsim
-f1\_Arith.s -da0 -pf0

Biggest used address for Text Section (word aligned): 0x1c

Biggest used address for Data Section (word aligned): 0x0

(dlxsim)

\begin{enumerate}
\def\labelenumi{\arabic{enumi}.}
\setcounter{enumi}{5}
\item
  Then in dlxsim you can use ``go'' or ``step'' command to simulate all
  instructions or each instruction step by step respectively.
\end{enumerate}

(dlxsim) step

stopped after (single) step, pc = \_main+0x04 (0x0004): addi r2,r0,0x9

(dlxsim) step

stopped after (single) step, pc = \_main+0x08 (0x0008): or r3,r1,r2

(dlxsim) get r2

r2: 0x00000009

(dlxsim) go

TRAP \#0 received

Altogether 41,0e0(41) cycles executed.

0 Warnings for unresolved data dependencies printed.

0 Warnings for successive load/store commands printed.

0 Warnings for load/stores in the text section printed.

(dlxsim)

\begin{enumerate}
\def\labelenumi{\arabic{enumi}.}
\setcounter{enumi}{6}
\item
  You can see different statistics using ``\emph{\textbf{stats}}''
  command.
\item
  Enter ``\emph{\textbf{quit}}'' command to exit from dlxsim simulator.
\end{enumerate}

\begin{enumerate}
\def\labelenumi{\Alph{enumi}.}
\setcounter{enumi}{1}
\item
  \textbf{Simulating a C file}
\end{enumerate}

\begin{enumerate}
\def\labelenumi{\arabic{enumi}.}
\setcounter{enumi}{8}
\item
  If the application consists of C files then you can use
  ``\emph{\textbf{make sim}}'', which will compile your application into
  assembly file and automatically starts dlxsim. The other steps remain
  the same. Remember, ``\emph{\textbf{make sim}}'' only works if you
  have already created Compiler.
\end{enumerate}

asip04@i80labpc04:\textasciitilde/ASIP\_SS17/Session1/ASIPMeisterProjects/brownie/Applications/Arith:\$make
sim

\begin{enumerate}
\def\labelenumi{\arabic{enumi}.}
\setcounter{enumi}{9}
\item
  You can have different parameter to ``\emph{\textbf{make sim}}'' like
  optimization identifier and number of NOPS added for simulating your
  application in hardware.
\end{enumerate}

asip04@i80labpc04:\textasciitilde/ASIP\_SS17/Session1/ASIPMeisterProjects/brownie/Applications/Arith:\$make
sim GCC\_PARAM=-O3

\begin{enumerate}
\def\labelenumi{\arabic{enumi}.}
\setcounter{enumi}{10}
\item
  You can now start dlxsim simulation using following different
  commands:
\end{enumerate}

asip04@i80labpc04:\textasciitilde/ASIP\_SS17/Session1/ASIPMeisterProjects/brownie/Applications/Arith:\$make
dlxsim GCC\_PARAM=-O3

OR

asip04@i80labpc04:\textasciitilde/ASIP\_SS17/Session1/ASIPMeisterProjects/brownie/Applications/Arith:\$make
dlxsim DLXSIM\_PARAM="-fBUILD\_SIM/Arith.dlxsim -da0 --pf1"

\begin{enumerate}
\def\labelenumi{\arabic{enumi}.}
\setcounter{enumi}{11}
\item
  You can save dlxsim simulation output to different file using
  ``\emph{\textbf{--lf}}'', ``\emph{\textbf{--uf}}'', or
  \emph{\textbf{``--af}}'' for LCD, UART or audio respectively as
  following:
\end{enumerate}

asip04@i80labpc04:\textasciitilde/ASIP\_SS17/Session1/ASIPMeisterProjects/brownie/Applications/Arith:\$make
dlxsim DLXSIM\_PARAM="-fBUILD\_SIM/Arith.dlxsim -da\textbf{0}
--pf\textbf{1} --lf\textbf{output\_dlxsim.txt}"

\end{document}
