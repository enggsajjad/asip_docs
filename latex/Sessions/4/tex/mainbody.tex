\hypertarget{synthesis-and-hardware-implementation}{%
\chapter*{Synthesis and Hardware Implementation}\label{synthesis-and-hardware-implementation}}

\rightline{\textbf{{1 Week}}}

\section*{Motivation and introduction}

In this session we synthesis our basis CPU with Xilinx ISE and execute a
test application on real hardware. The synthesis reports tell us how
much area and power is consumed by our CPU and what is the critical path
of our design. In this session, we will compile a C-code application
direct its output the LCD and UART with the help of some predefined
libraries. This session also introduces about different peripheral where
we forward our text/data, and how different libraries are used for
different peripherals. For every part, that starts like ``a)'', ``b)''
\ldots{} you have to mail the answers and asked files/tables to
\textbf{sajjad.hussain@kit.edu} and use the topic ``asipXX-Session3'',
with XX replaced by your group number.

\section*{Exercises}

\begin{enumerate}
\item \textbf{Preparing your project}
	\begin{enumerate}
		\item
		You can use the same project as in the last session, and just create
		a separate application subdirectory the application. You can start a
		fresh project as in the Session 1, but this would be time consuming.
		\item
		For the C application, you have to create two subdirectories in the
		``\emph{Application}'' directory e.g. ``\emph{Hello\_SW}'' and
		``\emph{Hello\_HW}''. Copy your application from
		``\emph{/home/ces-asip00/Sessions/Session4/app.c}'' to these. This is a
		simple example to direct a text to some peripheral devices like LCD
		or UART. ``\emph{Hello\_SW}'' is aimed for dlxsim and ModelSim
		simulations and ``\emph{Hello\_HW}'' is aimed for real hardware
		implementation.
		\item
		Copy a ``\emph{Makefile}'' file from the ``\emph{TestPrint}''
		application subdirectory to each application subdirectory.
		\item
		Set proper parameters and settings in ``\emph{env\_settings}.
		\item
		For these exercises, we will be using pipeline forwarding (-pf1)
		option which is the default one.
		\item
		Make sure that you already have VHDL files and GNU tools in your
		project's meister directory.
	\end{enumerate}
\item \textbf{Compiling and ModelSim Simulation}
	\begin{enumerate}
		\item
		First, you have to compile the application using gcc compiler to
		compare with the later results from dlxsim and ModelSim. For
		\emph{gcc} you can forward the printed output to a file, e.g.
		``\emph{a.out \textgreater{} output\_gcc.txt}'' (`\emph{a.out}' is
		the default name of the binary that is created when you compile
		``\emph{gcc arrayloop.c}'' while `\emph{output\_gcc.txt}' then
		contains the printed array). To compile with GCC, comment the line
		``\emph{\#define ASIP}''.
		\item
		However, for compiling it, you first need to provide the required
		libraries from /home/ces-asip00/epp/StdLib to your respective
		application, i.e. copy ``\emph{lib\_lcd\_dlxsim.c}'',
		``\emph{lib\_uart.c}'', ``\emph{loadStoreByte.c}'',
		``\emph{string.c}'' and respective header files to ``Hello\_SW''
		directory. Also, copy ``\emph{lib\_lcd\_320.c}'',
		``\emph{lib\_uart.c}'', ``\emph{loadStoreByte.c}'',
		``\emph{string.c}'' and respective header files to ``Hello\_HW''
		directory.
		\item
		Go to your application subdirectory ``Hello\_SW'', and type
		``\emph{\textbf{make clean}}'' clean this directory it there are
		previously generated files.
		\item
		In the directory ``Hello\_SW'', compile the C application using
		``\emph{\textbf{make sim}}''. A directory
		``\emph{\textbf{BUILD\_SIM}}'' is created which contains different
		temporary files and a .dlxsim file to be simulated in dlxsim. In
		this directory, the files ``\emph{\textbf{TestData.IM}}'' and
		``\emph{\textbf{TestData.DM}}''are the file used during the ModelSim
		simulation.
		\item
		Simulate your application in dlxsim simulator using
		``\emph{\textbf{make dlxsim}}'', just to verify the functionality.
		For dlxsim you can forward the LCD/UART output to a file, using the
		``-lf'' and ``-uf'' parameters respectively, e.g. ``make dlxsim
		DLXSIM\_PARAM=''-da0 --pf1 -lflcd.out -ufuart.out'' writes output to
		the file ``lcd.out'' and ``uart.out'' in the application directory.
		\item
		In your project directory, go to the ``ModelSim'' directory and
		start the ModelSim using ``vsim''. You can use the previous ModelSim
		project and simulate the application. Remember to generate VCD files
		required for power estimation while doing ModelSim simulation.
		\item
		After compiling, simulate the application in dlxsim and ModelSim and
		compare whether the printed results are the same as expected. The
		dlxsim and ModelSim will print text to a virtual LCD/UART. While
		ModelSim automatically writes to the file ``lcd.out'' and
		``uart.out''.
		\begin{enumerate}[label=(\alph*)]
			\color{red}\item\normalcolor
			How many cycles are required to execute this program DLXsim and
			ModelSim?
		\end{enumerate}
	\end{enumerate}
\item \textbf{Xilinx ISE Framework for Hardware Implementation}
	\begin{enumerate}
		\item
		Go to your application subdirectory ``Hello\_HW'', and type
		``\emph{\textbf{make clean}}'' clean this directory it there are
		previously generated files.
		\item
		In the directory ``Hello\_HW'', compile the C application using
		``\emph{\textbf{make sim}}''.
		\item
		Go to the project directory and type ``ise \&'' to start Xilinx ISE.
		\item
		Create new project using File Menu \textgreater{} New Project with
		following project settings:
	\begin{lstlisting}
Project Name: ISE_Framework
Project Path: PATH_TO\_YOUR_PROJECT/ISE_Framework
Device Family: Virtex5
Device: xc5vlx110t
Package: ff1136
	\end{lstlisting}
	\item
	Add the design and framework files by selecting ``Project Menu
	\textgreater{} Add Copy of Sources'' then brows to:
	\begin{enumerate}
		\item
		``PATH\_TO\_YOUR\_PROJECT\emph{\textbf{/ ISE\_Framework}}'' and
		select all the files
		\item
		``PATH\_TO\_YOUR\_PROJECT\emph{\textbf{/ ISE\_Framework/IP-Cores}}''
		and select all the files
		\item
		``PATH\_TO\_YOUR\_PROJECT\emph{\textbf{/
				meister/}}\emph{\textbf{browstd32.syn}}'' and select all the files
	\end{enumerate}
	\item
	Select top level modules ``\textbf{dlx\_toplevel}'', and now you can
	synthesize, implement and generate programming file for the design
	using the following respectively:
	\begin{enumerate}
		\item
		Processes Menu \textgreater{} Synthesize XST
		\item
		Processes Menu \textgreater{} Implement Design
		\item
		Processes Menu \textgreater{} Generate Programming File
	\end{enumerate}
	\item
	Once the design is implemented you can see different reports using:
	\begin{enumerate}
		\item
		Processes Menu \textgreater{} Place \& Route \textgreater{} Generate
		Post Place \& Route Static Timing \textgreater{} Detailed Reports
		\textgreater{} Place and Route Report
		\item
		Processes Menu \textgreater{} Place \& Route \textgreater{} Generate
		Post Place \& Route Static Timing \textgreater{} Detailed Reports
		\textgreater{} Post PAR Static Timing Report
		\item
		Processes Menu \textgreater{} Place \& Route \textgreater{} Analyze
		Post Place \& Route Static Timing \textgreater{} Timing Constraints
	\end{enumerate}
	\item
	In the project directory and type ``hterm \&'' to start HyperTerminal
	to see the UART output if there is any output. and adjust its settings
	like: Baud rate=115200, Stop bit=1, Data bits=8, Parity=None, COM
	Port=ttyUSB0 (for example), Newline at=CR+LF,
	\item
	In the application subdirectory and type ``make fpga'', it will
	combine the generate DM/IM file with your ISE generated bitstream.
	Finally, a new bitstream file containing your hardware CPU along with
	corresponding IM/DM files of your application will be generated in the
	folder ``BUILD\_FPGA''. This bitstream will be used to configure the
	FPGA.
	\item
	For hardware implementation you need to connect to i83labpc10 only.
	Connect your FPGA to PC the i83labpc10, power the board. In the
	application subdirectory type ``make upload'': to upload the existing
	bitstream to the FPGA
	\item
	\color{blue}If your application does not work try to RESET the FPGA/LCD board.\normalcolor
	\begin{enumerate}[label=(\alph*),start=2]
		\color{red}\item\normalcolor
		Does it work on FPGA? Do you see anything on LCD/UART?
		ModelSim?
	\end{enumerate}
\end{enumerate}
\item \textbf{Xilinx ISE Framework for Benchmarking}
	\begin{enumerate}
		\item
		To accurately measure the critical path and area of the ASIPmeister
		CPU, you can use ISE\_Benchmark folder instead of ISE\_Framework
		folder.
		\item
		Go to the project directory and type ``ise \&'' to start Xilinx ISE.
		\item
		Create new project using File Menu \textgreater{} New Project with
		following project settings:
	\begin{lstlisting}
Project Name: ISE_BenchMark
Project Path: PATH_TO_YOUR_PROJECT/ISE_BenchMark
Device Family: Virtex5
Device: xc5vlx110t
Package: ff1136
	\end{lstlisting}
	\item
	Add the design and framework files by selecting ``Project Menu
	\textgreater{} Add Copy of Sources'' then brows to:
	\begin{enumerate}
		\item
		``PATH\_TO\_YOUR\_PROJECT/ ISE\_ BenchMark'' and select all the
		files
		\item
		``PATH\_TO\_YOUR\_PROJECT\emph{\textbf{/ meister/ browstd32.syn}}''
		and select all the files
	\end{enumerate}
	\item
	Now you can synthesize, implement and generate programming file for
	the design as before.
	\item
	Once the design is implemented you can see different reports as
	before.
	\begin{enumerate}[label=(\alph*),start=3]
	\color{red}\item\normalcolor
	Reports: P\&R Report \textgreater{} Check for "Device Utilization
	Summary" to see \#Slices and \#LUT consumed. How many Slices and LUT are being used by the design?
	\color{red}\item\normalcolor
	Post PAR Static Timing Report: Check for "Timing Summary" to see the
	minimum period and maximum frequency supported by the processor
	architecture. What is the critical path and corresponding maximum attainable frequency?
	\end{enumerate}
\end{enumerate}
\item \textbf{Xilinx ISE Framework for XPower Power Estimation}
	\begin{enumerate}
		\item
		To accurately measure the power consumption of the ASIPmeister CPU,
		you can create another folder ISE\_XPower.
		\item
		Go to the project directory and type ``ise \&'' to start Xilinx ISE.
		\item
		Create new project using File Menu \textgreater{} New Project with
		following project settings:
	\begin{lstlisting}
Project Name: ISE_XPower
Project Path: PATH_TO_YOUR_PROJECT/ISE_XPower
Device Family: Virtex5
Device: xc5vlx110t
Package: ff1136
	\end{lstlisting}
	\item
	Add only design files by selecting ``Project Menu \textgreater{} Add
	Copy of Sources'' then brows to
	``PATH\_TO\_YOUR\_PROJECT\emph{\textbf{/ browstd32.syn}}'' and select
	all the files.
	\item
	Now you can synthesize and implement the design as before.
	\item
	Once the design is implemented you can open XPower tool using
	Processes Menu \textgreater{} Place \& Route \textgreater{} Analyze
	Power Distribution (xPower Analyzer)
	\item
	Then in XPower Tool, select ``File Menu \textgreater{} Open Design''
	and set the properties as follows:
	\begin{enumerate}
		\def\labelenumii{\arabic{enumii}.}
		\item
		Design File: PATH\_TO\_YOUR\_PROJECT/ISE\_ XPower/ BrownieSTD32.ncd
		\item
		Physical Constraint File: PATH\_TO\_YOUR\_PROJECT/ ISE\_ XPower/
		BrownieSTD32.pcf
		\item
		Simulation Activity File: PATH\_TO\_YOUR\_PROJECT/test.vcd
	\end{enumerate}
	\item
	After analyzing the activity file, the CPU power is estimated. You can
	see total and dynamic power of the FPGA. In addition, you can confirm
	that the VCD file is loaded properly by verify the clock value in
	XPower.
	\begin{enumerate}[label=(\alph*),start=5]
	\color{red}\item\normalcolor
	What is the total power consumed? What is the distribution of power
	i.e., dynamic and static power?
	\color{red}\item\normalcolor
	Why static/quiescent/leakage power is so high?
	\color{red}\item\normalcolor
	What is the power distribution among On-Chip \textbf{1}) Clocks,
	\textbf{2}) Logic, \textbf{3}) Signals, \textbf{4}) BRAMS, \textbf{5})
	IOs?
	\end{enumerate}
\end{enumerate}
\end{enumerate}
\textbf{Next Session:} Adding Custom Instructions

\textbf{Readings for the next session}:

Laboratory Chapters 8.2.3, 3.2.2, 3.2.3, 4.4,

ASIPmeister Tutorial

ASIPmeister User Manual
