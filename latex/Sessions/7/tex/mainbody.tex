\hypertarget{bubble-sort-power-area-estimation-and-hardware-implementation}{%
\chapter*{Bubble Sort -- Power \& Area Estimation and Hardware Implementation}\label{bubble-sort-power-area-estimation-and-hardware-implementation}}

\rightline{\textbf{{1 Week}}}

\section*{Motivation and introduction}

In this exercise, you will synthesize and implement the bubblesort
application and then download it to FPGA board and see the results on
the UART terminal or LCD. For visualizing, the output of BubbleSort and
some additional information is printed to the URAT interface. You can
use t\_print() for directing output to LCD or u\_print() to UART.
Remember, you need to add respective libraries. Using these frameworks,
the Bubble sort algorithm which will be implemented using the two CPUs
to form two versions:

\textbf{Version1:} basis CPU (\emph{browstd32.pdb})

\textbf{Version2:} optimized CPU (\emph{browstd32opt.pdb}) which
supports new instructions

For every part, that starts like ``a)'', ``b)'' \ldots{} you have to
mail the answers and asked files to \textbf{sajjad.hussain@kit.edu} and
use the topic ``asipXX-Session6'', with XX replaced by your group
number.

\section*{Exercises}
\begin{enumerate}
\item \textbf{Preparing and Simulating Version 1}
	\begin{enumerate}
		\item
		You have to create the software and the hardware sub directories
		under ``\emph{Application}'' for browstd32.pdb CPU. First, create a
		new project directory inside your \emph{ASIPMeisterProjects}
		directory for the new CPU and name it ``\emph{browstd32''}. You can
		use a copy from the \emph{browstd32} CPU project from the last
		session, but do not forget to adjust the ``\emph{env\_settings''}.
		Remember, for LCD interface you need to create two separate
		subdirectories for the software and the hardware application in
		``Applications'' directory.
		\item
		Copy the provided and ``\emph{app\_UART.c''} from
		``/home/asip00/Sessions/Session7'' to the created LCD or UART
		subdirectories respectively. This file directs the printing to UART,
		you can change it to LCD by replacing u\_print() to t\_print().
		However, UART interfacing is sufficient.
		\item
		Copy the C libraries from ``\emph{asip00/epp/StdLib}'' to each
		application subdirectory. For LCD simulation in dlxsim and ModelSim,
		use ``lib\_lcd\_dlxim.c''. For real LCD implementation in FPGA use
		``lib\_lcd\_320.c''.
		\item
		Also, copy the ``\emph{Makefile}'' to both the subdirectories.
		\item
		Make sure that you already have generated the VHDL files and GNU
		Tools for your CPU.
		\item
		Compile (``\emph{make sim}'') the application for basis CPU and
		generates the required .dlxsim and DM/IM file for the dlxsim and
		ModelSim respectively.
		\item
		Simulate the application with dlxsim using ``\emph{make dlxsim
			DLXSIM\_PARAM=''-da0 --pf1 --ufBubbleUART.out''}''. It will start
		the dlx simulator to simulate the compiled file generated in the
		previous stage. Here, you have to pass some parameters to dlxsim
		such as the LCD/UART file to print the outputs.
		\item
		You can restart the CPU by pressing the ``\emph{reset}'' push button
		on the small mini board on the FPGA board, but REMEMBER, that your
		array in data memory is already sorted after the first run, so the
		second, third \ldots{} run will be significantly faster than the
		first one.
		\item
		The BubbleSort framework is measuring the number of cycles for the
		execution of the bubbleSort methods. This measurement is done by a
		counter on the FPGA Board or in dlxsim/ModelSim respectively. This
		measurement only measures the bubbleSort method, but not the
		overhead for e.g. printing the result.
	\end{enumerate}
\item \textbf{Implementing the Project}
	\begin{enumerate}
		\item
		Create your ISE project as discussed in the session 4. Synthesize,
		implement and generate the bitfiles.
		\item
		Then from the application direcotory run ``make fpga'' and ``make
		upload''.
		\item
		You can check the results on the UART. You can open HyperTerminal
		using ``hterm \&''.
		\begin{enumerate}[label=(\alph*),start=1]
			\color{red}\item\normalcolor
			If your design works correctly, find out the design statistics
			(critical path, maximum frequency and area)
			\color{red}\item\normalcolor
			Compute the accurate time (in ms) required to sort the 20 numbers. Use
			the number of executed cycles (printed on the URAT interface) and the
			max. CPU frequency on the FPGA board, where the sorting is still
			correct).
			\color{red}\item\normalcolor
			Analyse the time and find the critical path (see Chapter 6.5 of the
			Laboratory Script)
		\end{enumerate}
	\end{enumerate}
\item \textbf{Power Estimation}
	\begin{enumerate}
		\item
		During ModelSim simulation also generate the VCD files for mentioned
		frequencies (first with 50MHz and then with Max. Frequency found in
		the last session).
		\item
		Create Xilinx ISE project to estimate power with XPower.
		\begin{enumerate}[label=(\alph*),start=4]
			\color{red}\item\normalcolor
			Determine the total and dynamic power.
			\color{red}\item\normalcolor
			Compute the total execution time (ms). You can use execution as the \#
			of cycles multiplied by the clock cycle in ModelSim.
			\color{red}\item\normalcolor
			Compute the energy required. Fill in all these results in the table
			below e.g. \emph{PowerReport.xlx} or \emph{PowerReport.ods}.
			\color{red}\item\normalcolor
			Does using any instruction minimize the required energy? A version
			uses an application, which needs less number of clock cycles than
			another Version; is it also power and/or energy-optimized version
			compared to Version2?
			\color{red}\item\normalcolor
			Repeat a-d, but instead of taking the default of 50 MHz, use the
			individual maximum CPU frequency on which a CPU can run (You can get
			it from ISE\_Benchmark). This frequency has to be configured in
			tb\_brownie32std.vhd (search for CLK\_PERIOD; e.g. 10 ns half period =
			20 ns period = 50 MHz). XPower will automatically load this frequency
			from the VCD file.
		\end{enumerate}
	\end{enumerate}
\item \textbf{Preparing, Simulating, Implementing and Power Estimation for Version2}
	\begin{enumerate}
		\item
		Repeat the above exercises for the optimized version a), b), c), d), e).
		\item
		Show the power consumption report as Sample PowerReport shown in the following table.
	\end{enumerate}
\end{enumerate}

% Please add the following required packages to your document preamble:
% \usepackage[table,xcdraw]{xcolor}
% If you use beamer only pass "xcolor=table" option, i.e. \documentclass[xcolor=table]{beamer}
\begin{table}[!htb]
	\centering
\begin{tabular}{|l|l|l|l|l|}
	\hline
	\multicolumn{1}{|c|}{\textbf{Versions}}                                                           & \multicolumn{1}{c|}{\cellcolor[HTML]{F1F1F1}\begin{tabular}[c]{@{}c@{}}Total   Power \\ {[}mW{]}\end{tabular}} & \multicolumn{1}{c|}{\cellcolor[HTML]{F1F1F1}\begin{tabular}[c]{@{}c@{}}Dynamic   Power \\ {[}mW{]}\end{tabular}} & \multicolumn{1}{c|}{\cellcolor[HTML]{F1F1F1}\begin{tabular}[c]{@{}c@{}}Execution   Time \\ {[}ms{]}\end{tabular}} & \multicolumn{1}{c|}{\cellcolor[HTML]{F1F1F1}\begin{tabular}[c]{@{}c@{}}Energy  \\  {[}nJ{]}\end{tabular}} \\ \hline
	\cellcolor[HTML]{F1F1F1}\begin{tabular}[c]{@{}l@{}}Version1   \\ 50 MHz\end{tabular}              &                                                                                                                &                                                                                                                  &                                                                                                                   &                                                                                                           \\ \hline
	\cellcolor[HTML]{F1F1F1}\begin{tabular}[c]{@{}l@{}}Version2   \\ 50 MHz\end{tabular}              &                                                                                                                &                                                                                                                  &                                                                                                                   &                                                                                                           \\ \hline
	\cellcolor[HTML]{F1F1F1}\begin{tabular}[c]{@{}l@{}}Version1 \\ Max. Freq:   -----MHz\end{tabular} &                                                                                                                &                                                                                                                  &                                                                                                                   &                                                                                                           \\ \hline
	\cellcolor[HTML]{F1F1F1}\begin{tabular}[c]{@{}l@{}}Version2 \\ Max. Freq: -----MHz\end{tabular}   &                                                                                                                &                                                                                                                  &                                                                                                                   &                                                                                                           \\ \hline
\end{tabular}
\end{table}

\textbf{Next Session:} Adaptive Differential Pulse Code Modulation
(ADPCM)

\textbf{Readings:} Recall relevant information from Laboratory Script,
ASIPmeister Tutorial and User Manual
