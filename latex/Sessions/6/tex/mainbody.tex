\hypertarget{bubble-sort-simulation-optimisation}{%
\chapter*{Bubble Sort -- Simulation \& Optimisation}\label{bubble-sort-simulation-optimisation}}

\rightline{\textbf{{1 Week}}}

\section*{Motivation and introduction}

In this session, we will start applying the whole design flow to a
\emph{BubbleSort} algorithm. You will receive the C code for
\emph{BubbleSort} to be simulated. Afterwards you will optimize its
performance by adding new instruction(s). In a later session, we will
estimate what we have paid for this speedup; in terms of chip area,
power and energy consumption, i.e. we will compare the basis processor
with the modified/extended one. Finally, in a later session, we will
implement both processors on the FPGA board. For every part, that starts
like ``a)'', ``b)'' \ldots{} you have to mail the answers and asked
files to \textbf{sajjad.hussain@kit.edu} and use the topic
``asipXX-Session5'', with XX replaced by your group number.

\section*{Exercices}

\begin{enumerate}
\item \textbf{BubbleSort Algorithm}
	\begin{enumerate}
		\item
		Have a look at ``\emph{BubbleSort\_Index.c''}. Every part, which
		contains a \emph{printf} function call, is encapsulated with a
		\emph{``\#ifndef ASIP''} directive. The reason is, that the
		\emph{printf} function usually is resolved to an operating system
		call (managing the screen and other resources), but for our CPU we
		don't have an operating system, thus we ignore the \emph{printf}
		function for our simulations. For hardware execution in a later
		session, we will map this call to a UART terminal or LCD. For a
		\emph{gcc} compiled version the \emph{printf} is a helpful in
		debugging the output.
		\item
		Look at the implementation of the algorithm. You will need a good
		knowledge of the algorithm for later optimizations. Compile
		\emph{``BubbleSort\_Index.c''} with ``\emph{gcc BubbleSort\_Index.c
			--o BubbleSort\_Index}'', look at the printed output when executing
		the binary and understand how the algorithm is working by going
		through the printed output gradually.
		\begin{enumerate}[label=(\alph*)]
			\color{red}\item\normalcolor
			How often the code of the inner loop is executed (not only the
			exchange part, the whole inner loop)? Please do not only answer this
			question, but also go through the output step by step. First, you
			should look at the code and think about the answer and then you should
			add a counter to the code to compute the correct result just to make
			sure, your prediction is correct.
		\end{enumerate}
		\item
		To simulate \emph{BubbleSort} with \emph{dlxsim} and \emph{ModelSim}
		it has to be translated from C to assembly, which will be done in
		the later exercises. To make the translation easier, the
		``\emph{BubbleSort\_Address.c''} has been prepared. Compile
		``\emph{BubbleSort\_Address.c''} with \emph{gcc} as discussed
		before.
		\item
		First, make sure that the output of the \emph{gcc} compiled versions
		of ``\emph{BubbleSort\_Index.c''} and
		``\emph{BubbleSort\_Address.c''} is the same. Then have a more
		detailed look into the address-version. The main difference between
		both versions is the way of accessing the array. The index-version
		uses an indexed access (e.g. array {[}j+1{]}). This usually
		translates into a chain of assembly instructions. First, the real
		address has to be computed and then the value can be loaded. The
		real address is: ``starting address from array'' + ``size of one
		array entry'' * ``index (i.e. j+1)''. In the inner loop of
		\emph{BubbleSort} we traverse through the array linearly, so we do
		not have to compute the real address every time from the scratch,
		instead we can just update the last computed real address. Two other
		changes against the index-version are, that every memory access is
		explicitly written, like ``\emph{value\_j = *j;}'' and the number of
		memory accesses is optimized as compared to the index-version.
		\begin{enumerate}[label=(\alph*),start=2]
			\color{red}\item\normalcolor
			How many load- and how many store- instructions are executed for each
			inner loop (distinguish between when there is exchange and no
			exchange)? Compare the index-version against the address-version and
			mention the two main points, why the address-version needs less memory
			accesses.
		\end{enumerate}
	\end{enumerate}
\item \textbf{Preparing your project}
	\begin{enumerate}
		\item
		You can use the same project as in the first session, and just
		create a separate application subdirectory for each example. You can
		start a fresh project as in the Session 1, but this would be time
		consuming.
		\item
		For the C application, you have to create subdirectory in the
		``\emph{Application}'' directory (e.g. ``sorting''), and copy your
		application from ``\emph{/home/ces-asip00/Sessions/Session6/bubble.c}''
		to here.
		\item
		Copy a ``\emph{Makefile}'' file from the ``\emph{TestPrint}''
		application subdirectory to each application subdirectory.
		\item
		Set proper parameters and settings in ``\emph{env\_settings}.
		\item
		For these exercises, we will be using pipeline forwarding (-pf1)
		option which is the default one.
		\item
		Make sure that you already have VHDL files and GNU tools in your
		project's meister directory.
	\end{enumerate}
\item \textbf{Compiling the application in dlxsim and ModelSim with basis processor}
	\begin{enumerate}
		\item
		Now, compile ``bubble.c'' using ``make sim''.
		\item
		After compiling, simulate the application in \emph{dlxsim} and
		\emph{ModelSim} and compare whether the printed results are the same
		compared to a \emph{gcc}-compiled version. For dlxsim, you can use
		the command ``get \_array 20d'', it should give you the sorted
		array. For ModelSim, you can see the sorted array in the
		TestData.OUT file.
		\begin{enumerate}[label=(\alph*),start=3]
			\color{red}\item\normalcolor
			How many cycles do you need for execution in dlxsim and ModelSim?
		\end{enumerate}
	\end{enumerate}
\item \textbf{Bubble Sort -- Optimisation: Customizing the basis processor}
	\begin{enumerate}
		\item
		Create a project directory for this session by copying the directory
		``\emph{/home/ces-asip00/­epp/ASIP­Meister­Projects/TEMPLATE\_PROJECT/}''
		and renaming it (e.g. \emph{brownieOPT}).
		\item
		Now we start optimizing our bubblesort application for speed. There
		might be two options for you.
		\begin{enumerate}
			\item
			Create a new application sub-directory (e.g. ``sortingOptS'') and
			copy ``sorting/BUILD\_SIM/bubble.s'' from the last exercise to
			this directory and start optimizing the code. In the assembly
			code, look for different possibility to define custom instructions
			and replace that part of code with the custom instruction in
			assemble file. If you start with assembly file, sometime it gives
			errors for labels that starts with dot. Change it to dashes. like
			.L6 to \_L6.
			\item
			Create a new application sub-directory (e.g. ``sortingOptC'') and
			copy ``bubble.c'' to this directory and start optimizing the code.
			You can directly look into the ``bubble\emph{.c}'' and define some
			custom instruction to replace some part of the code with new
			custom instruction.
		\end{enumerate}
		\item
		However, before optimizing, create a simple C or assembly file to
		test different custom instruction (e.g. OPT) into an application
		subdirectory i.e. \emph{TestOPT}.
		\item
		Copy a ``\emph{Makefile}'' file from the ``\emph{TestPrint}''
		application subdirectory to each application subdirectory.
		\item
		Copy the provided \emph{ASIPMeister} CPU file
		``\emph{browstd32.pdb''} from ``/home/ces-asip00//Sessions/Session1/''
		into your project directory, and rename it
		``\emph{browstd32OPT.pdb''}
		\item
		Set proper parameters and settings in ``\emph{env\_settings}'' as
		discussed in Figure 2-5 in the Laboratory Script. Specially the
		followings:
\begin{lstlisting}
export PROJECT_NAME=brownieOPT
export CPU_NAME=browstd32OPT
export ASIPMEISTER_PROJECTS_DIR=${HOME}/ASIPMeisterProjects
export DLXSIM_DIR=/home/ces-asip00/epp/dlxsimbr_Laboratory
\end{lstlisting}
	\end{enumerate}
\item \textbf{Adding the new instruction to \emph{ASIPMeister}}
	\begin{enumerate}
		\def\labelenumii{\arabic{enumii}.}
		\item
		Now we start adding new instructions to our processor to speed up
		the execution.
		\item
		In your project directory start \emph{ASIPMeister} and add the new
		instruction to your new CPU. First, define a new instruction format
		for your instruction if it does not match with the existing
		instruction formats.
		\item
		Use the available opcode.
		\item
		You also have to define ``CKF Prototype'' for each new custom instruction in ASIPmeister, generate GNU tools and use inline assembly in C code.
		\item
		Add the custom instructions to dlxsim as well.
		\item
		Write a small C or assembly code to test your new instruction.
		\item
		Generate the hardware and software files from ASIPMeister and
		simulate the new instruction with \emph{ModelSim}. Use the small
		test application that you created to test your \emph{dlxsim}
		implementation in the previous exercises for this purpose.
		\item
		If everything is working fine, then simulate the \emph{BubbleSort}
		C/Assembly code that uses the new instruction in \emph{ModelSim}.
		\item
		Compile the optimized bubblesort application using ``\emph{make
			sim}'' and then ``\emph{make dlxsim}''. Make sure, that the
		resulting array is still correct.
		\begin{enumerate}[label=(\alph*),start=4]
			\color{red}\item\normalcolor
			How many cycles do you need for execution?
			\color{red}\item\normalcolor
			What is the speedup compared to \emph{original code} (i.e. \#Cycles
			without custom instruction / \#Cycles with custom instruction)?
			\color{red}\item\normalcolor
			Attach the assembly and/or C file you used to test the custom
			instruction and to test the given application.
			\color{red}\item\normalcolor
			Attach the modified ASIPmeister .pdb file.
		\end{enumerate}
	\end{enumerate}
\end{enumerate}

\textbf{Next Session:} Bubble Sort - \textbf{Power \&} Area

\textbf{Estimation and} Hardware Implementation

\textbf{Readings for the next session}: Laboratory Chapters 6 \& 7
