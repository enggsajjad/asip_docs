\hypertarget{assembly-programs}{%
\chapter*{Assembly Programs}\label{assembly-programs}}

\rightline{\textbf{{1 Week}}}

\section*{Motivation and Introduction}

The main goal of this session is to get used to assembly programs. Every
example shows a basic operation. Examine the code, simulate it with
\emph{dlxsim} and answer the corresponding questions. The assembly code
is available in the directory
``\emph{/home/asip00/Sessions/Session2/''}. \emph{Dlxsim} is available
in the directory
``\emph{asip00/}\emph{epp/dlxsimbr\_}\emph{Laboratory/'' (default)}. You
can copy \emph{dlxsim} to your directory or you can start it from the
default directory. For every exercise, the part that starts like ``a)'',
``b)'' \ldots{} you have to write an answer. This answer should mainly
prove that you have understood the problem, so you can make your answers
short, but they still have to answer everything, that was asked. Mail
your answer to \textbf{sajjad.hussain@kit.edu} and use the topic
``asipXX-Session1'', with XX replaced by your group number.

\section*{Exercises}

The following exercise correspond to the given assembly files in the
``\emph{/home/asip00/Sessions/Session1}'' directory.

\begin{enumerate}[start=0]
\item \textbf{Preparing your project}
	\begin{enumerate}
		\item
		You can use the same project as in the last session, and just create
		a separate application subdirectory for each assembly example. You
		can start a fresh project as in the Session 1, but this would be
		time consuming.
		\item
		For each application (C or Assembly), you have to create
		subdirectories in the ``\emph{Application}'' directory.
		\item
		Copy a ``\emph{Makefile}'' file from the ``\emph{TestPrint}''
		subdirectory to each application subdirectory.
		\item
		Set proper parameters and settings in ``\emph{env\_settings}.
		\item
		For these exercises, we will run basic assembly programs in dlxsim
		and see the effect of pipeline forwarding (-pf1) or no forwarding
		(-pf0), like
	\end{enumerate}
\begin{lstlisting}
	make dlxsim DLXSIM_PARAM="-da0 -pf0"
\end{lstlisting}
\item \textbf{Basic Assembly Instructions}
	\begin{enumerate}
	\item Understand the functionality of every instruction and understand the
	values of every target register after the program run has completed (see
	the dlxsim chapter for reading register values in dlxsim simulation).
	Check the number of cycles needed to execute all instructions.
	\begin{enumerate}[label=(\alph*)]
	\color{red}\item\normalcolor What is the reason for this high number of cycles? Which instruction
	causes that behaviour and why is it doing so?
	\color{red}\item\normalcolor What is the code/data address space? What is starting and ending
	address of code/data section?
	\end{enumerate}
	\end{enumerate}
\item \textbf{Memory Access}
	\begin{enumerate}[label=(\alph*)]
	\color{red}\item\normalcolor
	Explain the goal of the instruction combination \emph{LSOI /ADDI}.
	What is generally (so: not only for this specific example) the
	register value after \emph{ADDI}, what is it after the additional
	\emph{LSOI}?
	\color{red}\item\normalcolor
	Why is it in general not possible to omit the \emph{LSOI} instruction,
	although it would be possible in this special example?
	\color{red}\item\normalcolor
	Read about ``forwarding'' and ``load delay slot'' in brownie32
	datasheet. Disable NOP after the load instruction, and try executing
	with --pf0 and --pf1 both.
	\end{enumerate}
\item \textbf{Branches}
	\begin{enumerate}[label=(\alph*)]
	\color{red}\item\normalcolor
	Which high-level control structure (e.g. `call subroutine' \ldots) is
	implemented in this example code?
	\color{red}\item\normalcolor
	What is computed with this example? R24 = \emph{function} (R21, R22);
	\color{red}\item\normalcolor
	Look at the \emph{NOP} instructions and explain why they are placed
	there.
	\end{enumerate}
\item \textbf{Loops}
	\begin{enumerate}[label=(\alph*)]
	\color{red}\item\normalcolor 
	What is computed with this example? R23 = \emph{function} (R21, R22).
	Debug the application step-by-step with the capabilities of dlxsim.
	\color{red}\item\normalcolor
	The approach to compute the function in the way this example is doing
	it has two specific names. Do you know those names? (One is founded by
	the main operations, while the other is founded historically. You
	either know the names or you don't. If you don't know both names, you
	may guess)
	\color{red}\item\normalcolor
	In general, how often is the loop maximally executed? How the input
	data has to look like to get this maximal number of iterations?
	\color{red}\item\normalcolor
	Enable and disable the first NOP, and try executing with --pf0 and
	--pf1 both.
	\end{enumerate}
\item \textbf{A High Level Structure}
	\begin{enumerate}[label=(\alph*)]
	\color{red}\item\normalcolor
	Which high-level control structure do you recognize? Explain the
	purpose of the instruction-block between ``\emph{ADDI R23, R0,
		\$(2)}'' and ``\emph{JPR R24}''.
	\color{red}\item\normalcolor
	Why do you have to shift by value 3? Explain it with a close view to
	the body of the control structure and pay attention to the addressing
	mode of the DLX processor.
	\color{red}\item\normalcolor
	What are the general differences between branch and jump instructions
	in the DLX instruction set (also have a look at the different
	instruction formats to find a part of the answer)?
	\end{enumerate}
\item \textbf{ASM Directive in C}
	\begin{enumerate}
	\item You can use assemble instruction directly in c as follows {[}1,2{]}:
	\begin{lstlisting}
int @\aftergroup\redcolor@res@\aftergroup\blackcolor@=0;
int @\aftergroup\bluecolor@in1@\aftergroup\blackcolor@=20;
int @\aftergroup\bluecolor@in2@\aftergroup\blackcolor@=30;
int main() {
	__asm__ volatile (
	"add 	%[@\aftergroup\greencolor@out@\aftergroup\blackcolor@], 	%[@\aftergroup\greencolor@op1@\aftergroup\blackcolor@], 	%[@\aftergroup\greencolor@op2@\aftergroup\blackcolor@]\n"
	: [@\aftergroup\greencolor@out@\aftergroup\blackcolor@] "=&r" (@\aftergroup\redcolor@res@\aftergroup\blackcolor@)
	: [@\aftergroup\greencolor@op1@\aftergroup\blackcolor@] "r" (@\aftergroup\bluecolor@in1@\aftergroup\blackcolor@),[@\aftergroup\greencolor@op2@\aftergroup\blackcolor@] "r" (@\aftergroup\bluecolor@in2@\aftergroup\blackcolor@)
	);
	return 0;
}
	\end{lstlisting}
	\item
	In any asm block, assembly instructions appear first, followed by the
	inputs and outputs, which are separated by a colon. The assembly
	instructions can consist of one or more quoted strings. The first
	colon separates the output operands; the second colon separates the
	input operands. If there are clobbered registers, they are inserted
	after the third colon. If there are no clobbered inputs for the asm
	block, the third colon can be omitted, as Listing 2 shows.
	\begin{lstlisting}
int A[10] = {45,23,0,0,0,0,0,0,0,0};
int main() {
	__asm__volatile (
	"add 	%[@\aftergroup\greencolor@out@\aftergroup\blackcolor@], 	%[@\aftergroup\greencolor@op1@\aftergroup\blackcolor@], 	%[@\aftergroup\greencolor@op2@\aftergroup\blackcolor@]	\n"
	: [@\aftergroup\greencolor@out@\aftergroup\blackcolor@] "=&r" (A[2])
	: [@\aftergroup\greencolor@op1@\aftergroup\blackcolor@] "r" (A[0]),	[@\aftergroup\greencolor@op2@\aftergroup\blackcolor@] "r" (A[1])
	);
	return 0;
}
	\end{lstlisting}
	\item There is no output operand for the sw instruction, hence the outputs section of the asm is empty. None of the registers is modified, so they are all input operands, and the target address is passed in with
	the input operands. However, something is modified: the addressed
	memory location.
	\begin{lstlisting}
int @\aftergroup\redcolor@res@\aftergroup\blackcolor@ [] = {0,0,0,0,0,0,0,0};
int @\aftergroup\bluecolor@a@\aftergroup\blackcolor@=45;
int main() {
	
	__asm__ volatile  ( 
	" sw    %@\aftergroup\greencolor@1@\aftergroup\blackcolor@(%@\aftergroup\greencolor@2@\aftergroup\blackcolor@),	%@\aftergroup\greencolor@0@\aftergroup\blackcolor@        \n"
	:
	: "r"(@\aftergroup\bluecolor@a@\aftergroup\blackcolor@), "i"(sizeof(int)),"r"(&@\aftergroup\redcolor@res@\aftergroup\blackcolor@)
	);
	
	return 0;
}
	\end{lstlisting}
	\item
	Branching can be tricky with inline asm. Using labels, the branch-to
	address can be designated with a unique identifier that can be used as
	a target branch address.
	\begin{lstlisting}
int @\aftergroup\redcolor@x, y, z@\aftergroup\blackcolor@;
int @\aftergroup\bluecolor@a=45, b=23, c=1@\aftergroup\blackcolor@;
int main() {
__asm__ volatile ( 
"  		addi 	%[@\aftergroup\greencolor@out1@\aftergroup\blackcolor@], 	%[@\aftergroup\greencolor@op1@\aftergroup\blackcolor@], 	%[@\aftergroup\greencolor@op2@\aftergroup\blackcolor@]                    \n"     
" 	 	brnz     	%[@\aftergroup\greencolor@op3@\aftergroup\blackcolor@],	here                          	\n"         
"  there: 	add 	%[@\aftergroup\greencolor@out2@\aftergroup\blackcolor@], %[@\aftergroup\greencolor@op1@\aftergroup\blackcolor@], 	%[@\aftergroup\greencolor@op2@\aftergroup\blackcolor@]                    \n"       
"  here:  	mul 	%[@\aftergroup\greencolor@out3@\aftergroup\blackcolor@], %[@\aftergroup\greencolor@op1@\aftergroup\blackcolor@], 	%[@\aftergroup\greencolor@op2@\aftergroup\blackcolor@]                   	\n"         
: [@\aftergroup\greencolor@out1@\aftergroup\blackcolor@] "=&r" (@\aftergroup\redcolor@x@\aftergroup\blackcolor@),	[@\aftergroup\greencolor@out2@\aftergroup\blackcolor@] "=&r" (@\aftergroup\redcolor@y@\aftergroup\blackcolor@), 	[@\aftergroup\greencolor@out3@\aftergroup\blackcolor@] "=&r" (@\aftergroup\redcolor@z@\aftergroup\blackcolor@)                                    
: [@\aftergroup\greencolor@op1@\aftergroup\blackcolor@] "r" (@\aftergroup\bluecolor@a@\aftergroup\blackcolor@),	[@\aftergroup\greencolor@op2@\aftergroup\blackcolor@] "r" (@\aftergroup\bluecolor@b@\aftergroup\blackcolor@) ,	[@\aftergroup\greencolor@op3@\aftergroup\blackcolor@] "r" (@\aftergroup\bluecolor@c@\aftergroup\blackcolor@)
:  "r0" );
	return 0;
}
	\end{lstlisting}
	\item See ``A guide to inline assembly for C and C++'' for further details.
	\begin{enumerate}[label=(\alph*)]
	\color{red}\item\normalcolor
	Write a C program using assembly instructions to load two consecutive
	values from an array and store their addition or subtraction at the
	fourth location of an array depending on the third value of array i.e.
	0 means addition and 1 means subtraction.
	\end{enumerate}
\end{enumerate}
\end{enumerate}

\textbf{Next Session:} ModelSim Simulation

\textbf{Readings for the next session}: Chapters 2.3, 4 \& 5

\textbf{{[}1{]}.
	\url{https://gcc.gnu.org/onlinedocs/gcc/Extended-Asm.html}}

\textbf{{[}2{]}.\url{https://dmalcolm.fedorapeople.org/gcc/2015-08-31/rst-experiment/how-to-use-inline-assembly-language-in-c-code.html}}
