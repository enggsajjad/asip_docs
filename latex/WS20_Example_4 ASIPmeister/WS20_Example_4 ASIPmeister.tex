% Options for packages loaded elsewhere
\PassOptionsToPackage{unicode}{hyperref}
\PassOptionsToPackage{hyphens}{url}
%
\documentclass[
]{article}
\usepackage{amsmath,amssymb}
\usepackage{lmodern}
\usepackage{iftex}
\ifPDFTeX
  \usepackage[T1]{fontenc}
  \usepackage[utf8]{inputenc}
  \usepackage{textcomp} % provide euro and other symbols
\else % if luatex or xetex
  \usepackage{unicode-math}
  \defaultfontfeatures{Scale=MatchLowercase}
  \defaultfontfeatures[\rmfamily]{Ligatures=TeX,Scale=1}
\fi
% Use upquote if available, for straight quotes in verbatim environments
\IfFileExists{upquote.sty}{\usepackage{upquote}}{}
\IfFileExists{microtype.sty}{% use microtype if available
  \usepackage[]{microtype}
  \UseMicrotypeSet[protrusion]{basicmath} % disable protrusion for tt fonts
}{}
\makeatletter
\@ifundefined{KOMAClassName}{% if non-KOMA class
  \IfFileExists{parskip.sty}{%
    \usepackage{parskip}
  }{% else
    \setlength{\parindent}{0pt}
    \setlength{\parskip}{6pt plus 2pt minus 1pt}}
}{% if KOMA class
  \KOMAoptions{parskip=half}}
\makeatother
\usepackage{xcolor}
\IfFileExists{xurl.sty}{\usepackage{xurl}}{} % add URL line breaks if available
\IfFileExists{bookmark.sty}{\usepackage{bookmark}}{\usepackage{hyperref}}
\hypersetup{
  hidelinks,
  pdfcreator={LaTeX via pandoc}}
\urlstyle{same} % disable monospaced font for URLs
\setlength{\emergencystretch}{3em} % prevent overfull lines
\providecommand{\tightlist}{%
  \setlength{\itemsep}{0pt}\setlength{\parskip}{0pt}}
\setcounter{secnumdepth}{-\maxdimen} % remove section numbering
\ifLuaTeX
  \usepackage{selnolig}  % disable illegal ligatures
\fi

\author{}
\date{}

\begin{document}

\textbf{ASIPMEISTER -TUTORIAL}

\begin{enumerate}
\def\labelenumi{\arabic{enumi}.}
\item
  Login to any \emph{\textbf{i80labpcXX.ira.uka.de}} directly or using
  SSH or using X2Go Client. For example, login as
  \emph{\textbf{asip-sajjad04}} into
  \emph{\textbf{i80labpc02.ira.uka.de}}
\item
  Open shell terminal from the start menu. It should be in your default
  home directory. Type ``\emph{\textbf{pwd}}''
\end{enumerate}

asip04@i80labpc04:\textasciitilde:\$pwd

/home/asip04

\begin{enumerate}
\def\labelenumi{\arabic{enumi}.}
\setcounter{enumi}{2}
\item
  Create a new directory for your Lab e.g.
  ``\emph{\textbf{ASIP\_SS17''}}
\end{enumerate}

asip04@i80labpc04:\textasciitilde:\$mkdir ASIP\_SS17

asip04@i80labpc04:\textasciitilde:\$cd ASIP\_SS17/

\begin{enumerate}
\def\labelenumi{\arabic{enumi}.}
\setcounter{enumi}{3}
\item
  Create a new directory for your lab session in
  ``\emph{\textbf{ASIP\_SS17''}} e.g. ``\emph{\textbf{Session1''}}
\end{enumerate}

asip04@i80labpc04:\textasciitilde/ASIP\_SS17:\$mkdir Session1

asip04@i80labpc04:\textasciitilde/ASIP\_SS17:\$cd Session1/

\begin{enumerate}
\def\labelenumi{\arabic{enumi}.}
\setcounter{enumi}{4}
\item
  Create a directory for your ASIP project in
  ``\emph{\textbf{Session1''}}
\end{enumerate}

asip04@i80labpc04:\textasciitilde/ASIP\_SS17/Session1:\$mkdir
ASIPMeisterProjects

asip04@i80labpc04:\textasciitilde/ASIP\_SS17/Session1:\$cd
ASIPMeisterProjects/

\begin{enumerate}
\def\labelenumi{\arabic{enumi}.}
\setcounter{enumi}{5}
\item
  For each ASIPmeister CPU create a separate in
  ``\emph{\textbf{ASIPMeisterProjects''}}. For example, copy template
  project and rename it e.g. ``\emph{\textbf{brownie''}} for ASIPmeister
  CPU ``\emph{\textbf{brownie.pdb}}''
\end{enumerate}

asip04@i80labpc04:\textasciitilde/ASIP\_SS17/Session1/ASIPMeisterProjects:\$cp
-r /home/asip00/epp/ASIPMeisterProjects/TEMPLATE\_PROJECT ./brownie

asip04@i80labpc04:\textasciitilde/ASIP\_SS17/Session1/ASIPMeisterProjects:\$ls

brownie

\begin{enumerate}
\def\labelenumi{\arabic{enumi}.}
\setcounter{enumi}{6}
\item
  Set the parameters and settings of the ASIP project in
  ``\emph{\textbf{env\_settings}}''
\item
  Open ASIPMeister CPU in the respective directory i.e. in brownie
\end{enumerate}

asip04@i80labpc04:\textasciitilde/ASIP\_SS17/Session1/ASIPMeisterProjects/brownie:\$ASIPmeister
brownie.pdb \&

\begin{enumerate}
\def\labelenumi{\arabic{enumi}.}
\setcounter{enumi}{8}
\item
  Modify the CPU in ASIPmeister and generate the required files. A
  ``\emph{\textbf{meister}}'' directory will be created in your ASIP
  project directory i.e. in ``\emph{\textbf{brownie}}''
\item
  For another CPU copy template project and rename it e.g.
  ``\emph{\textbf{brownieOPT''}} for ASIPmeister CPU
  ``\emph{\textbf{brownieSPEED.pdb}}''
\end{enumerate}

asip04@i80labpc04:\textasciitilde/ASIP\_SS17/Session1/ASIPMeisterProjects:\$cp
-r /home/asip00/ASIPMeisterProjects/TEMPLATE\_PROJECT ./brownieOPT

asip04@i80labpc04:\textasciitilde/ASIP\_SS17/Session1/ASIPMeisterProjects:\$ls

brownie brownieOPT

\begin{enumerate}
\def\labelenumi{\arabic{enumi}.}
\setcounter{enumi}{10}
\item
  Set the parameters and settings of the ASIP project in
  ``\emph{\textbf{env\_settings}}''
\item
  Open ASIPMeister CPU in the respective directory i.e. in brownieOPT
\end{enumerate}

asip04@i80labpc04:\textasciitilde/ASIP\_SS17/Session1/ASIPMeisterProjects/brownieOPT:\$ASIPmeister
brownieSPEED.pdb \&

\begin{enumerate}
\def\labelenumi{\arabic{enumi}.}
\setcounter{enumi}{12}
\item
  Modify the CPU in ASIPmeister and generate the required files. A
  ``\emph{\textbf{meister}}'' directory will be created in your ASIP
  project directory i.e. in ``\emph{\textbf{brownieOPT}}''
\end{enumerate}

\end{document}
